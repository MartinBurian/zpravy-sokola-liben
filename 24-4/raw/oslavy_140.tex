OSLAVY 140 LET SOKOLA LIBEŇ

Rozkaz (sám sobě) zněl jasně: „Napiš nějaké shrnutí oslav!{}``

\ldots oslavy, 140 let\ldots{} -- ale jak vlastně začít? A jak psát, aby
to vůbec někdo četl\ldots{}

Minimalisticky?: Oslavy \st{140 let Sokola Libeň} proběhly\ldots{}

Telegraficky?: Oslavy Sokola Libeň -- 140 let -- bohatý program -
\ldots{}

Formálně?: Výbor jednoty se dne x. x. 2023 usnesl, že přípravou oslav je
pověřen\ldots{}

Začít od konce?: Nenene, to nejlepší až na konec, protože jinak nebudou
číst ten zbytek\ldots{} Ale co když to až do konce nikdo nedočte?

Už to mám -- strukturovaně! -- Každý si přečte, co ho zajímá\ldots{}
Snaž se, chlapče, třeba to dočtou!

Tedy od začátku.

Sokol Libeň byl oficiálně založen 26. října 1884. Za svou dlouhou
140letou historii zažil již mnohé, ale vždy se po časech nepřízně znovu
zvedl a neúnavně sledoval svůj sokolský „směr a cíl``.

Sokolem Libeň prošly již celé generace cvičenců, pro mnohé je budova
sokolovny druhým domovem, v~němž se pohybují spřízněné duše s~podobným
smýšlením. Jsem přesvědčen, že Sokol Libeň může být přeneseně vnímán
jako živý organismus, jako jedna velká vícegenerační rodina, čítající
dnes asi 800 členů, neboť základem každé rodiny jsou lidé. Lidé, kterým
není líto investovat nezištně svůj čas do společné „věci``, předávat své
zkušenosti a um mladší generaci a věřit, že právě ti mladší budou
ochotni se stejným nasazením a ochotou převzít otěže společného díla a
rozvíjet všestrannou činnost důstojně dál.

Rodinná rada (čti výbor jednoty) proto již v~roce 2023 rozhodla
uspořádat oslavy ku příležitosti 140. výročí, jejichž program postupně
krystalizoval.

\textbf{Logo oslav}

Nejdříve byla na sklonku roku 2023 vyhlášena soutěž o grafický návrh
loga oslav. Vítězná skica znázorňující siluetu naší secesní sokolovny
nás tedy provází celými oslavami. Mimo jiné je i základem pro muší
křídla, která jsme nechali pro potřeby oslav vyrobit.

\includegraphics[width=3.91144in,height=0.90931in]{media/image1.png}

\textbf{Články do Osmičky}

Vzhledem k~tomu, že Sokol Libeň představuje v~rámci Prahy 8 významnou
kulturně-sportovní instituci, seznamovali jsme v~sérii článků
v~měsíčníku Osmička širší veřejnost s~fungováním jednoty a představili
nejvýznamnější oddíly a pořádané akce.

\textbf{Prapor mužů}

Náčelník jednoty Josef Kubišta (Pepišta) pojal myšlenku, že naší sbírce
praporů něco chybí. Ano, byl to prapor -- prapor mužů, na nějž se
zejména muži složili a ku příležitosti slavnostní tělocvičné akademie a
10. výročí obnovení oddílu mužů jej předali jednotě.

\hl{FOTKA PRAPORU}

\textbf{Fotografie na galerii}

Sokol Libeň je již několik let objektem zájmu fotografa Honzy
Jirkovského. S~jeho pomocí bylo vybráno 16 černobílých fotografií
s~tělocvičnou tématikou, které nyní zdobí galerii naší sokolovny.
Výstava je doplněna o medailon autora. Nutno dodat, že všechny
fotografie byly pořízeny přímo v~sokolovně, a pokud jste je ještě
neviděli, doporučujeme tyto umělecké kousky Vaší ctěné pozornosti.

\includegraphics[width=1.86038in,height=1.37252in]{media/image2.png}

\textbf{Almanach}

Uvedené výročí nás také přinutilo oprášit a doplnit almanach vydaný ke
120. narozeninám jednoty. Tento nesnadný a časově náročný úkol přijal na
svá bedra jednatel jednoty bratr Jan Přech (Řek), jemuž po grafické
stránce sekundoval Martin Burian, korektorka Martina Waclawičová a další
dobrovolníci.

Křídový papír dodává textu i fotografiím punc důstojnosti, kterou si
vskutku zaslouží. Posuďte sami! Alamanach je vám k~dispozici za
dobrovolný příspěvek 200 Kč ve vrátnici sokolovny. Platba možná hotově
nebo za pomoci QR kódu.

\textbf{Meotarové vystoupení}

Hlavní program oslav, plánovaných na víkend 22.--24. listopadu 2024, byl
zahájen vtipnou exkurzí do historie Sokola Libeň pod názvem „Obrazy z
dějin Sokola libeňského aneb nevážná meotarová rozpustilost ve věci
dějepravy``. Slovem nás provázel začínající „scénárista`` Tomáš Troup,
kresby na meotar pokládal jejich autor, občasný karikaturista Jáchym
Kaplan (Jack). Nutno zdůraznit, že neotřelý koncept s~využitím meotaru
sklidil u diváků nebývalý ohlas. Bylo tedy dle programu rozhodnuto, že
po tak úspěšné premiéře je nutné představení opakovat. A tak se skutečně
konala tentýž den i derniéra uvedeného kusu. Pokud se posouváním po
meotaru obrázky příliš „neotřely``, možná dostanou příležitost i ti,
kteří svou první a „poslední`` šanci v~pátek 22. listopadu 2024
propásli.

Z~řad obecenstva dokonce zaznívaly tak troufalé návrhy, že by se
představení mělo postoupit široké veřejnosti například v~nedaleké
knihovně u Löwita. Tvůrci však mají obavu, aby je nesemlela tamní
nesokolská kritika.

\hl{FOTO MEOTAR}

\textbf{Výstava}

Po prvním meotarovém vystoupení se valná většina přítomných přesunula do
Alšova sálu na vernisáž výstavy. I když sálu vévodí skica, jejímž
autorem je slavný malíř Mikoláš Aleš, hlavním předmětem zájmu byla
tentokrát výstava, která pod nepřekvapivým názvem „140 let Sokola
Libeň`` umožnila návštěvníkům nahlédnout pod pokličku naší historie a
prohlédnout si alespoň zlomek hodnotných artefaktů, které se k~naší a
sokolské historii pojí. Výstavu připravila dlouholetá vzdělavatelka
jednoty Anna Holanová (Anka), která v~archivu bedlivě střeží všechny
naše duchovní poklady.

V~průběhu víkendového trvání výstavy bylo zajímavé sledovat, jak se
mnozí z~návštěvníků poznávali na dobových fotografiích z~90. let i na
těch aktuálnějších; jak starší vyprávěli mladším, co všechno zažili;
s~úctou procházeli kolem praporů a nejstarších předmětů; připomínali si
bývalé činovníky jednoty a poznávali na fotografiích ty současné;
vzpomínali na jednotlivé všesokolské slety\ldots{} V~neděli po poledni
jsme však naše „rodinné album`` zase uschovali do archivu, kde počká i
s~ostatními předměty na další vhodnou příležitost.

\hl{FOTO VÝSTAVA}

\textbf{Tělocvičná akademie}

O této události se zcela jistě dočtete i na jiných místech Zpráv. Abych
však sledoval nit událostí, dovolím si o ní pojednat v tomto textu.

Slavnostní tělocvičná akademie se pod taktovkou náčelníka Josefa Kubišty
(Pepišty) konala v~sobotu 23. listopadu 2024 od 16 h. Svou návštěvou
sokolovnu k~této příležitosti poctilo i mnoho významných hostů včetně
starosty městské části Praha 8 p. Ondřeje Grose a starosty České obce
sokolské br. Martina Chlumského.

Při slavnostním nástupu byla sbírka jednotových praporů doplněna o
prapor mužských složek, který tímto muži věnovali jednotě. Prapor za
doprovodu Foerstrova komorního pěveckého sdružení pokřtila emeritní
členka Sokola Libeň sestra Marie Kselíková. Slavnostní atmosféra byla
dokreslena i zdařilým barevným osvětlením.

O výbornou atmosféru se po dobu celé akademie starali cvičenci, kteří
předvedli mnoho všestrannostně zaměřených výstupů, i přihlížející
diváci. Cvičební program doplnil i „remake`` veleúspěšné scénky „Cvičení
na koni``. Nostalgicky jsme také zavzpomínali na mnohé sletové skladby
-- vždyť v~Libni se nacvičovaly téměř všechny\ldots{} Sami diváci svým
potleskem a nadšením dokázali, že se i přes bohatý program čítající
celkem 21 čísel skvěle bavili.

\hl{FOTO AKADEMIE}

\textbf{Loutkové divadlo}

Sokol Libeň provozuje již několik let i vlastní ochotnické loutkové
divadlo. Když byl principál Tomáš Troup požádán, zda by soubor v~rámci
oslav nesehrál v~neděli dopoledne pro děti loutkové představení, ochotně
souhlasil a sáhl po (již několikrát) osvědčeném kusu „Tři zlaté vlasy
děda Vševěda``. Nevěřící Tomáš byl však od počátku na pochybách, kolik
se tak asi může v~neděli dopoledne sejít diváků\ldots{} Principálovy
pochyby rozptýlily před desátou hodinou až zástupy dětí, které plny
očekávání vlekly své rodiče na zmíněné loutkové představení.

Jistě víte, že zejména mužské složky hoví přesným statistikám. Proto Vám
můžeme přesně sdělit, že diváků, včetně těch dospělých, se nakonec sešlo
nepočítaně, což je opravdu mnoho\ldots{}

Skromným odhadem shlédlo představení asi 80 diváků, kteří zaplnili téměř
celý Srncův sál, jenž sloužil pro potřeby loutkového divadla již ve 20.
letech 20. století.

Dopoledne patřilo dětským divákům, zatímco ti dospělí, jak jste možná
z~několika narážek odhadli, teprve čekali na zlatý hřeb večera.

\textbf{České nebe v divadle Járy Cimrmana}

Vrcholem oslav 140. výročí byla totiž nedělní návštěva představení České
nebe v~prostorách Žižkovského divadla Járy Cimrmana. Pozitivní,
slavnostní a téměř rodinná atmosféra opanovala divadlo již před začátkem
představení. Úderem 19. hodiny se v~sále setmělo a v~zákulisí začali
nervózně přešlapovat řečníci.

Krátké úvodní slovo pronesl zasloužilý starosta jednoty Jiří Novák
(Jirkan), který krom jiného zmínil i důležitost duchovního rozměru
Sokola. Po starostovi, který dříve přiznal, že v~Sokole Libeň kroutí již
svou 51. sezónu, se na jeviště dostavil náčelník J. Kubišta. Bratr
náčelník však svůj proslov pojal zcela netradičně! Stěžejním prvkem jeho
statě byla analýza stavebního deníku naší sokolovny. Fundovaným a
precizně argumentovaným rozborem překvapil šokované publikum tím, že jej
přivítal, a zde budu raději citovat: „na prknech Sokola Libeň, v~Divadle
Járy Cimrmana!{}`` (Písemný záznam celého projevu máme k~dispozici a
budiž zachován pro věčnou památku této jedinečné návštěvy.)

Náčelníkovi se tak jeho fenomenální cimrmanovsky laděnou přednáškou
podařilo „zbořit`` diváky ještě dřív, než přišel sám „Cimrman``!

Náčelníka na jevišti vystřídaly cvičitelky oddílu akrojógy, které na
našich „libeňských`` prknech a za zvučného potlesku diváků předvedly své
nevšední akrojogínské umění.

Pak už patřila ta věhlasná prkna jen hercům. Jako první se na jevišti za
bouřlivého potlesku objevil Pan Zdeněk Svěrák (88 let), jenž přítomné
ujistil, že doposud „nevěděl, že hraje na prknech Sokola Libeň``\ldots{}
Jeho účinkování jakožto ikonické, nejen cimrmanovské osobnosti si
nesmírně vážíme; o to víc, že v~předvečer našeho představení se účastnil
charitativního večera StarDance pro Centrum Paraple! Skvělé bylo celé
představení, takže všech 238 libeňských diváků včetně čestných hostů
(Cimrman by je označil za místní honoraci) v~čele se starostou ČOS br.
M. Chlumským se celý večer výborně bavilo.

Sokol Libeň následně herce obdaroval symbolicky sokolskou kokardou a
výtiskem pamětního almanachu s věnováním.

Zmíněné představení bylo pomyslnou „pěknou tečkou za tou naší`` oslavou.
Pro znavené organizátory se tak stal tento večer motivací k~případným
dalším budoucím počinům podobného ražení.

Závěrem nezbývá než Sokolu Libeň do dalších mnoha let popřát spoustu
obětavých cvičitelů, nadšených dobrovolníků a odhodlaných následovníků,
abychom dokázali i v~dalších letech držet náš sokolský „směr a cíl``!

\hl{FOTKA DIVADLO}

Miloslav Doupal (mila.doupal@sokol-liben.cz)
