\documentclass[11pt]{article}

\usepackage[a5paper,left=2cm,right=1cm,top=1cm, bottom=1.5cm]{geometry}
\usepackage{array}
\usepackage{makecell}
\usepackage[all]{nowidow}
\usepackage{wrapfig}
\usepackage{enumitem}
\usepackage{multicol}

\usepackage{hyperref}
\hypersetup{
    colorlinks=true,
    urlcolor=sokolblue,
    }

\usepackage[czech]{babel}
\usepackage[utf8]{inputenc} 
\usepackage{ellipsis}

\usepackage{fontspec}
\newfontfamily{\tyrs}{Sokol Tyrs}
\newfontfamily{\fugner}{Sokol Fugner}

% \usepackage{lmodern}
% \usepackage[T1]{fontenc} 
\usepackage{anyfontsize}
\newcommand{\titlesize}{\fontsize{56pt}{67pt}}


\usepackage[dvipsnames]{xcolor}
\definecolor{sokolred}{RGB}{228, 5, 33}
\definecolor{sokoldarkred}{RGB}{200, 0, 30}
\definecolor{sokolblue}{RGB}{45, 46, 135}

\usepackage{tikz}
\usetikzlibrary{calc}

\usepackage{fancyhdr}

\fancypagestyle{standard}{%
    \fancyhf{}
    \fancyhead[LO]{%
        \begin{tikzpicture}[overlay,remember picture]
            \fill [color=sokolred] (current page.north west) rectangle ($ (current page.south west) + (1cm,0cm) $);
            \fill [color=sokolred] ($ (current page.north west) + (1.1cm,0cm) $) rectangle ($ (current page.south west) + (1.2cm,0cm) $);
        \end{tikzpicture}
        }
    % \fancyhead[RE]{%
    %     \begin{tikzpicture}[overlay,remember picture]
    %         \fill [color=orange](current page.north east) rectangle
    %             ($ (current page.south east) + (-1cm,0cm) $);
    %     \end{tikzpicture}
    %     }
    \fancyfoot[C]{%
      \begin{tikzpicture}[overlay,remember picture]
        \fill [color=sokolred] ($ (current page.south east) + (-1.5cm,1.3cm) $) rectangle ($ (current page.south east) + (0cm,0.5cm) $)
         node [pos=0.5,color=white] {\large\tyrs{\thepage}\hspace*{0.5cm}};
      \end{tikzpicture}
    }

    \renewcommand{\headrulewidth}{0pt}
    \renewcommand{\footrulewidth}{0pt}
}



\fancypagestyle{uvodnik}{%
    \fancyhf{}
    \fancyfoot[C]{%
      \begin{tikzpicture}[overlay,remember picture]
        \fill [color=sokolred] ($ (current page.south east) + (-1.5cm,1.3cm) $) rectangle ($ (current page.south east) + (0cm,0.5cm) $)
         node [pos=0.5,color=white] {\large\tyrs{\thepage}\hspace*{0.5cm}};
      \end{tikzpicture}
    }

    \renewcommand{\headrulewidth}{0pt}
    \renewcommand{\footrulewidth}{0pt}
}

\fancypagestyle{blank}{%
    \fancyhf{}
    \fancyfoot[C]{}

    \renewcommand{\headrulewidth}{0pt}
    \renewcommand{\footrulewidth}{0pt}
}


\newcommand{\post}[1]{%
\begin{center}
{\huge \tyrs #1}
\end{center}
}

\newcommand{\subpost}[1]{%
\vspace*{12pt}
\begin{center}
{\Large \tyrs #1}
\end{center}}

\newcommand{\signature}[2]{%
  \begin{flushright}
    \textbf{#1}\\#2
  \end{flushright}
}

\newcommand{\luv}{\clqq\kern-0.07em}
\newcommand{\ruv}{\kern0.07em\crqq\kern0.1em}

\usepackage{csquotes}
\DeclareQuoteAlias{german}{czech}
\MakeOuterQuote{"}

\usepackage[normalem]{ulem}


\begin{document}

%% title
\newgeometry{margin=1cm}
\pagecolor{sokolred}
\color{white}
\pagenumbering{gobble}
\begin{center}
\vspace*{\fill}

{\titlesize \fugner ZPRÁVY}

{\titlesize \tyrs SOKOLA LIBEŇ}

\vspace*{1cm}

{\large ročník XLIX · číslo 3 · září 2023}

\vspace*{\fill}
\end{center}

\clearpage
\normalcolor
\nopagecolor
\pagenumbering{arabic}

%% úvodník
\pagestyle{uvodnik}
\newgeometry{margin=1.5cm}


{\fontsize{48pt}{57pt} \fugner \color{sokolred} \noindent Úvodník}

\vspace*{12pt}

\noindent
Nazdar!

\vspace*{6pt}

\noindent
Když přišla Vítkova prosba na úvodník, ukázalo se, že by to mohl být i můj závěrečník (alespoň na nějakou dobu). Rodina se zvětšuje, děti naprosto nepřekvapivě vyžadují pozornost, kterou bych jim ráda dávala, a logistika přejíždění přes půl Prahy je neúprosná. Svoje náčelnické žezlo tak v~blízké době předám skvělému cvičiteli našich žákyň, Tomáši Dragounovi. Budiž mu Sokol výzvou i~oporou!

Teď trochu patosu –⁠ členem Sokola Libeň jsem díky rodičům od roku 1992 a od té doby jsem byla stále činným cvičencem a později i~cvičitelem. Prošla jsem za tu dobu spoustu oddílů, zažila nespočet výprav a táborů a v~Sokole si našla také spoustu přátel a kamarádů. Tyto sestry a bratry vnímám jako svoji druhou velkou rodinu. Díky cvičení a cvičitelských kurzům jsem se vlastně i tak nějak oklikou dostala ke své profesi –⁠ fyzioterapii. A~byla to dobrá volba, rozhodně cítím, že jsem na správném místě.

Věřím, že až mé děti odrostou, přibyde času (haha) a elánu se zas věnovat i dalším dětem a dospělým. Proto se rozhodně neloučím napořád a Sokol Libeň zůstává i nadále v~mé mysli. Sokolské cvičení (a Libeň!) doporučuji často i~v~ordinaci. A~to z~hlavního důvodu –⁠ jedná se o~všestranně rozvíjející primárně nevýkonnostní sport. Protože tlaku na výkon je všude plno, a ne každý ho snese. Ne každý je také na všechno stejně šikovný. Ale budit v~dětech touhu k~pohybu a pocit, že hýbat se je zábava, o~to jde. Protože pohyb je často právě spojen se zábavou, s~přáteli, se zdravým životním stylem. A~nemusí to být vždy ve formě honby po bodech, výsledcích a potažmo penězích.

Dosud jsem si díky Sokolu a pak i studiu vyzkoušela skoro všechny možné i nemožné sporty (jen tu tenisovou raketu jsem ještě nedržela, ale i to stále plánuji). V~ničem jsem nebyla nejlepší, nic jsem nedělala výkonnostně. Ale pro zdravé tělo, zdravý duch a dobrou kondici to úplně stačilo. Naopak cítím sílu právě v~tom rozhledu. Tím chci říct, že vést děti k~pohybové pestrosti a různosti je podle mě dobrá cesta.\linebreak
\clearpage\noindent
Získají tak velmi kvalitní fyzický i dovednostní základ tak, aby mohly jednou dělat, cokoliv budou chtít.

Všem cvičencům přeji kreativní cvičitele, cvičitelům pak děti plné nadšení.

\signature{Ája Krásová}{(dříve Duchačová)}

\clearpage

%% termínka

\pagecolor{sokolred}
\color{white}
\renewcommand{\arraystretch}{1.5}

\newcommand{\boxheight}{12cm}

\vspace*{\fill}
\post{TERMÍNOVÁ LISTINA AKCÍ JEDNOTY}
\vspace*{0pt}

\begin{center}
\begin{tikzpicture}
  \draw [ultra thick,color=white](0.3cm,0cm) rectangle (11.3cm,\boxheight);
  \fill [color=white] (0cm,0.3cm) rectangle (11cm,\boxheight + 0.3cm)
  node [pos=.5, color=black] {
    \begin{tabular}{l  p{6.5cm}}
      21. 9. 2023 (čt) & 23. ročník Běhu strmého do zámeckého vrchu \\
      24. 9. 2023 (ne) & Sletová štafeta (Sokol Hostivař; župní akce) \\
      5. 10. 2023 (čt) & Pietní akt k~Památnému dni sokolstva v~Libni \\
      7. 10. 2023 (so) & 65. Podzimní Výlet Libeňského Sokola \\
      10. 11. 2023 (pá) & plavecká štafeta o~pohár starostů pražských žup (děti i dospělí) \\
      11. 11. 2023 (so) & Podzimní brigáda v~sokolovně (zváni všichni členové) 9–15 h \\
      12. 11. 2023 (ne) & Župní sraz cvičitelů (župní akce) \\
      (25. 11. 2023) (ne) & Tělocvičná akademie Sokola Libeň\newline(předběžný termín! – alternativně v~lednu/únoru z~důvodu rekonstrukce šaten) \\
      7. 12. 2022 (čt) & Mikuláš \\
      21. 12. 2022 (čt) & Loutkáři (vánoční loutkové představení) a poslední cvičení v~roce 2022 \\
    \end{tabular}
  };
  %  node [color=black] {FOOBAR};
\end{tikzpicture}

\renewcommand{\arraystretch}{1}

\vspace*{12pt}

Sledujte naše stránky, kde budeme postupně uveřejňovat podrobnosti
k~jednotlivým akcím.

\end{center}
\vspace*{\fill}

\clearpage
\nopagecolor
\normalcolor

%% normální obsah
\restoregeometry
\pagestyle{standard}

\post{Letní dění mimo cvičení v~sokolovně}

Počet členů jednoty se nám ani po zvýšení příspěvků nesnížil a je nás cca 750 – vážím si toho.

\subpost{Hospodaření jednoty}
Letošní granty přinesly z~Národní sportovní agentury 373\,896\,Kč na podporu sportu dětí, vybavení a odměny cvičitelů (o~finance na odměny cvičitelů musíme žádat, protože na výši mezd cvičitelů je navázána výše podpory na vybavení). Od Magistrátu hlavního města Prahy jsme získali grant na provoz sportoviště ve výši 583\,000\,Kč a investiční grant na rekonstrukci šaten ve výši 5 997\,000\,Kč. Dále jsme od Prahy 8 obdrželi 700\,000\,Kč na podporu sportu dětí, 50\,000\,Kč na turistické oddíly a 35\,000\,Kč na loutkové divadlo. Bez investičního grantu na šatny je to 1 741 896\,Kč (loni bylo cca 2\,000\,000\,Kč). Takže jen drobný pokles.

V~červenci jsme podali daňové přiznání. Díky štědrým grantům, které sanovaly naše náklady, a poměrně slušným výnosům od nájemců nebytových prostorů byla naše daň poměrně vysoká – cca 225\,000\,Kč.

Od září se opět o~něco rozšiřuje pronájem tělocvičen pro okolní gymnázia a v~srpnu se v~sokolovně natáčel kousek filmu režiséra Stracha s~Jiřinou Bohdalovou – do pokladny nám zase přibylo pár desítek tisíc korun.

Trochu jsme také optimalizovali výdaje za energie. Po nabídce Pražské plynárenské, že nám sníží cenu o~11\% pod vládní limit jsme se rozhlédli u~konkurence a uzavřeli smlouvu s~firmou Tedom, kde je cena na 40\% vládního stropu. Jedná se o~měsíční fixace, ale snad se burza nepoblázní. U~nové firmy začínáme v~říjnu. U~stejné firmy budeme od ledna 2024 i s~elektřinou (cca 50\% vládního stropu).

\subpost{Opravy a zvelebování sokolovny}
Během prázdnin byla natřena okna u~nájemců Direkta a Gulag a momentálně se natírají okna v~místnosti bývalé matriky v~přízemí. Byla dokončena instalace zábradlí na nové loděnici na dvoře a začali jsme shromažďovat dokumenty k~její kolaudaci. Začala výroba a instalace nových úložných skříněk v~Srncově sále. Koncem prázdnin jsme připravili náhradní šatny náhradou za ty v~rekonstrukci.

Jako každý rok se už od jara pilně starám o~nechtěnou (plevel) i~žádanou (trávník) zeleň kolem celé sokolovny a snad je to trochu vidět. Navíc přibyla zatravněná střecha podzemní místnosti. A~vypiplat a udržet nově vysetou trávu bylo v~letošním horkém létě dost náročné.

\subpost{Šatny}
Hlavní událostí letošního roku je generální rekonstrukce šaten. Koncem května jsme se dozvěděli o~přidělení grantu a ihned jsme vypsali veřejnou zakázku na zhotovitele. Došlo 8 nabídek a vybrána byla nejlevnější nabídka firmy Wandel z~Ledče nad Sázavou (cca 10\,250\,000\,Kč vč. DPH). Mimochodem oproti loňské soutěži cena dosti výrazně klesla. Grantová smlouva byla s~magistrátem podepsána 20. června. Hned na začátku prázdnin nastoupilo do šaten komando cvičitelů (a cvičitelek, mužů, žactva) a mezi 3. a 7. červencem vystěhovalo na galerii všechny šatnové skříňky, vymontovalo z~šatnových klecí věšáky a zbouralo šatnové klece a odvezlo je do šrotu (vyneslo to asi 15\,000\,Kč). 

Taky jsme do nové loděnice přestěhovali věci ze staré garáže, kde bude mít stavební firma zázemí pro rekonstrukci šaten. Prostřídalo se 22 dobrovolníků. 

Po četných urgencích jsme konečně ze stavebního úřadu vylámali také prodloužení stavebního povolení (přišlo 18. července; žádost byla dána 22.~března; úřad na to má 30 dní, nám to trvalo 118 dní). 19.~července uplynuly také nutné lhůty z~veřejné zakázky, takže jsme ten den podepsali smlouvu se zhotovitelem a ještě týž den proběhlo předání staveniště a zahájení prací.

V~tuto chvíli jsou vybourané podlahy a staré příčky, nasypán štěrk a položen podkladní beton. Jsou vybourány všechny stavební otvory a postavena místnost nové šatny. Také se klubou nové kabinky WC ve sprchách. Zatím jde vše podle plánu. Tak si držme palce, ať se nic nezvrtne. Abychom těch skoro 6 milionů nemuseli vracet, musí být zkolaudováno do konce tohoto roku, tedy za necelé 4 měsíce.

Omlouváme se tedy cvičencům za snížený komfort v~náhradních šatnách a absenci sprch. Zato doufám, že se již v~lednu budeme převlékat ve zbrusu nových šatnách.

\clearpage
Do dalších měsíců si přejme co nejméně problémů, stabilizaci cen energií, příliv členů a co nejvíce dokončených vylepšení a oprav v~sokolovně.


\signature{Jiří Novák (Jirkan)}{starosta\\tel.: 602 284 198}

\vspace*{24pt}

\post{Zpráva místonáčelníka}
S~probíhající rekonstrukcí šaten je částečně nabourán běžný provoz sokolovny. To znamená, že od září do prosince nebudou k~dispozici šatny, ale ani sprchy a WC v~přízemí. K~dispozici jsou WC v~sále a na chodbě v~1. patře. Čeká nás tedy trocha improvizace a uskrovnění. Převlékat se budeme v~náhradních prostorách (místnost v~přízemí, sborovna, Filipova síň, balustráda, skříňky přemístěné ze šaten na galerii velkého sálu). Každému oddílu bude určen konkrétní prostor.

Dospělé členy prosíme, aby si s~sebou nosili svůj visací zámek a pro převlečení používali volné skříňky na galerii (a po cvičení je opouštěli otevřené, připravené pro použití dalšími cvičenci). \textbf{Omlouváme se za dočasně snížený komfort. Vlastní cvičení probíhá dle zvyklostí a nijak omezeno není.}

Více o~rekonstrukci v~samostatném článku br. starosty.

\subpost{Jaro a léto v~oddíle žáků a dorostenců}
Od 2. května až do prázdnin jsme cvičili venku a věnovali jsme se zejména atletickým disciplínám. 26.–28. května se konalo celostátní finále závodů ve všestrannosti mladšího žactva. Z~Libně postoupil Jiří Smutný a mezi 23 soupeři vybojoval úžasné 4. místo (9. v~gymnastice, 7. v~plavání, 3. v~atletice a 2. ve šplhu). V~soutěži družstev vybojovala naše župa (2 žáci + 2 žákyně) 5. místo ze 13 družstev. O~dva týdny později, 9.–11. 6., se konalo celostátní finále závodu ve všestrannosti staršího žactva, dorostu a dospělých. Z~Libně postoupilo 5 závodníků, dorazit mohli tři. Hugo Hőschl mezi staršími žáky obsadil 14. místo z~21, Václav Blahunek mezi dorostenci byl 17. z~20 a v~mužích obsadil Josef Kubišta 11. místo mezi 16 soupeři.

29. června jsme se za krásného počasí sešli u~ohně na dvoře sokolovny k~Zakončovacímu táboráku. Se členy dalších oddílů jednoty nás bylo asi 175 účastníků. Zpívaly se písničky, povídalo se o~dění v~uplynulém roce, opékali jsme buřty a pili šťávu. Žáci si po táboráku vyhlásili výsledky celoročních soutěží.

V~\textbf{Zimní soutěži 2023} mezi 55 \textbf{mladšími žáky} ve svých ročnících vyhráli: J. Cakl, O. Nekvapil, L. Bednář a F. Šefrna a v~celkovém pořadí se ještě dobře umístili: M. Trnka, M. Sokol, O. Boušek, J. Černý, M. Ptáček, R. Marin, M. Cakl, O. Ševčenko, O. Zarychnyi, F. Vinkler, A.~Holík, A. Kubát a V. Zubák.

Mezi 30 \textbf{staršími žáky} zvítězili ve svém ročníku: J. Smutný, V. Bőhm, Š. Novák, H. Hőschl a D. Vršecký a jmenování si dále zaslouží J. Komárek, O. Doupal, D. Pošta a J. Šefrna.

Mezi 19 \textbf{dorostenci a cvičiteli} byli nejlepší: M. Novotný, A. Novotný, T. Kléger, J. Šich, J. Pikálek, J. Novák, D. Unzeitig, P. Boháč, T. Novák a J. Kubišta. Celkem bylo hodnoceno 104 cvičenců.

V~\textbf{celoroční soutěži O~nejvěrnější docházku} byli nejpilnější tito cvičenci (přijít se mohlo 79x): 78x T. Kléger, 77x J. Kubišta a T.~Novák, 73x J. Novák, 72x J. Smutný a T. Novotný, 71x P. Boháč, 70x M. Sokol a F. Novák, 69x K. Vandas a Alexandre Basseville, 67x J. Doupal a J. Pikálek, 66x P. Ettel, 65x J. Černý, 64x T. Květoň, 62x O. Nekvapil, A. Musil a S. Sentűrk, 61x R. Hroz a O. Doupal, 59x J.~Vandasd a O. Ševčenko, 57x F. Šefrna, 56x R. Marin a J. Novák. 

Dalších 18 kluků mělo docházku mezi 50 a 70\%. Celkem bylo zapsáno 127 cvičenců (66 mladších žáků, 32 starších žáků, 13 dorostenců a 16 cvičitelů). Celoroční průměr docházky byl 52,53 při průměrně měsíčně zapsaných 100,7 cvičencích. 


\subpost{Dětský den}
Dětský den se letos konal 1. června v~obvyklém formátu. V~parku na louce u~studánky připravilo 53 našich pořadatelů 27 stanovišť, které si přišlo vyzkoušet 270 (!) dětí nejen z~našeho Sokola, ale i z~řad veřejnosti. Nechyběly ani tradiční vyjížďky na kánoích po přilehlé Vltavě a nakonec jako obvykle vystoupili šermíři ze skupiny Streitax. Počasí přálo a všechny děti si odnesly za své snažení kromě zážitků i věcné a sladké odměny.

\subpost{Letní tábory}
V~létě se naše turistické oddíly vydaly na své tábory do Tajanova u~Velhartic. 
Nejprve se o~víkendu 30. 6. – 2. 7. za účasti 40 (!) pracantů tábor za jediný víkend kompletně postavil. Museli jsme posekat a odvozit trávu z~louky, vykopat odpadovku a latrínu, postavit kuchyň, dřevník, seník, zásobák, stany a týpka, vztyčit stožár, postavit bránu a zábradlí a opravit sklípek a slavnostní ohniště.

První tentokráte na tábořišti pobyli jilmáci (30. 6. – 22. 7.). Bylo to 10 členů, u~kterých se prostřídalo 5 vedoucích. Jako druzí jeli bývalí členové (21.–29. 7.). Těch se postupně v~táboře objevilo 9 a spolu s~jejich rodinnými příslušníky se celkový počet účastníků vyšplhal na 20. Následovala Káňata (29. 7. – 12. 8.), přičemž se tábora zúčastnilo 21 dětí, u~kterých se prostřídalo 10 vedoucích. Jako poslední na tábor dorazily Veverky (12.–27. 8.), které měly 6 vedoucích a 9 členek. Celkem tedy 8 týdnů táborů s~81 táborníky. Počasí na začátku a konci horké, uprostřed studené a deštivé, ale zdravotních a jiných problémů minimum, zážitků spoustu, takže všichni můžeme na letošní tábory vzpomínat jen v~dobrém.

\subpost{Podzim v~oddíle žáků a dorostenců}
\textbf{Mladší žáci (2014–2017) cvičí v~út a čt v~17–18\,h, st. žáci a dorostenci (2006–2013) v~út a čt v~18–19\,h.} Pro cvičitele a dorostence je možnost úterního cvičení mužů a čtvrteční košíkové (vždy 19–20).

Pevně věříme, že kluci i díky Vám přivedou další nové kamarády. Docházka zatím sice vypadá slibně – na první hodině 5. září bylo 50~cvičenců, ve čtvrtek už 72 a z~nich 17 nových tváří, ale rádi uvítáme další zájemce o~cvičení (bereme mladší i starší žáky). Děkujeme a těšíme se. Doufáme, že v~nadcházejícím cvičebním roce se s~průměrem dostaneme nad 50.

V~září a začátkem října budeme cvičit opět venku (v~parku u~Rokytky a na zahradě sokolovny). \textbf{I~ven se nosí stejný cvičební úbor jako do tělocvičny} – modré trenky a bílé tričko se sokolským znakem. Na nohy je nutná vhodná zavazovací obuv (sandály opravdu nejsou atletické či fotbalové obutí). Jen v~případě deště a velmi chladného počasí se bude cvičení konat v~sokolovně. Nenavlékejte na kluky zbytečně tepláky a mikiny – pohybem se dostatečně zahřejí a drobné otužování upevní jejich zdraví.

\textbf{Zdůrazňujeme, že začátek i konec cvičení je v~šatně – děkujeme}. Pokračuje jarně-podzimní soutěž Letní disciplíny (atletika) a opět začíná celoroční soutěž O~nejvěrnější docházku s~pravidelným každoměsíčním vyhlašováním a rozdáváním diplomků za 100\% docházku.

Od 10. října budeme cvičit už jen v~sokolovně – kromě gymnastiky nás tedy čekají i vnitřní hry a též nácvik na Akademii. Připomínáme také, že za pozdní příchod či absenci úboru vybíráme pokutu 1 Kč (do fondu odměn za docházku). 

O~státním svátku \textbf{28. 9. žáci necvičí}, naopak o~podzimních prázdninách 26. 10. se cvičí.

O~kluky se bude starat náš mužský cvičitelský sbor v~tomto složení: Cvičitelé: J. Novák (52), T. Novák (50), J. Kudroň (34), J. Přibyl (31), J. Kubišta (30), J. Kerhart (20), P. Boháč (19), A. Novotný (19) a naši mladí pomahatelé: M. Novotný, J. Skokan, T. Kléger, V. Novák, V. Blahunek, J. Pikálek a A. Basseville. Od října, až se ustálí počty žáků a rozvrhy studujících cvičitelů, dostane každé družstvo žáků (děleno dle ročníků narození) své stálé cvičitele. V~říjnu vyšleme na školení cvičitelů další zájemce.

Výsledky závodů, fota, dopisy a další informace najdete na vývěskách v~sokolovně a na ní a také na www.sokol-liben.cz.

\subpost{Příspěvky}
\textbf{S~placením příspěvků prosím vyčkejte, dokud vám nepřijdou e-mailem platební instrukce ze systému EOS}. Bude to okolo 20.~září. Do té doby musíme žactvo přesunout do správného oddílu a zjistit, kolik hodin týdně chodí. Pokud přesto někomu přijde platba za špatný oddíl či za jiný počet hodin, neplaťte ji, přihlašte se do systému EOS, dejte „Nový požadavek na management“ a tam napište aktuální údaje. Původní platba bude stornována a přijde vám správná. Prosíme o~zaplacení příspěvků na druhé pololetí roku 2023 nejlépe do konce září a přes účet (žáci cvičící 1x týdně 1 275 Kč, žáci cvičící 2x týdně 1 650 Kč). 

Všechny členy, kteří již v~minulosti přihlášku za sebe či svoje děti vyplnili a odevzdali, prosíme, aby se zamysleli, jestli nedošlo k~nějaké změně (kontakty, zdravotní stav, …), a pokud ano, aby údaje v~systému EOS aktualizovali – děkujeme.

Zcela noví kluci se musí přihlásit – na stránkách dáte odkaz na systém EOS, kliknete na “nová registrace”, vyplníte základní údaje (jméno, datum narození, oddíl a počet hodin týdně) a odešlete. Systém vám do 24 hodin pošle kompletní přihlášku. Tu vyplníte a odešlete. V~řádu dní vám přijdou instrukce k~platbě příspěvků. Bližší informace k~vyplňování přihlášky naleznete v~návodu na našich internetových stránkách. 

V~matrice – 1. patro sokolovny (čtvrtek 15:45–18:45) lze zakoupit bílé tričko se znakem nebo znak na vlastní bílé tričko. 

\subpost{Nabídka dalších aktivit}
\textbf{Všem žákům nabízíme možnost účasti na výletech, které pořádají naše turistické oddíly}. Jilm a Veverky mají od letoška společný program a zvou starší žáky a žákyně. Schůzky jsou ve středu v~klubovně. Výpravy se konají cca 1x měsíčně. Vyvrcholením celoroční činnosti je pak letní tábor. V~roce 2024 to bude opět Tajanov u~Velhartic na Šumavě. Zatím se zdá, že se podaří udržet také činnost Káňat, kde se potýkáme s~nedostatkem vedoucích.

Líbí se vám \textbf{cvičení mužů} na akademiích? Klidně se můžete stát členem oddílu – berou další zájemce (muže 18–50 let). Cvičí se v~úterý v~19–20.

Výborně funguje také před několika lety založený \textbf{oddíl šplhu} (osmimetrové lano ze sedu bez přírazu). Řada jeho členů se účastní i mistrovství republiky v~tomto sportu. Též berou další zájemce od staršího žactva až po muže a od loňska šplhají i děvčata a ženy. 

Další větví v~mnohotvárné činnosti mužů je \textbf{Přetah lanem} – i v~tomto sportu se závodí a konají se i mezinárodní soutěže. 

\subpost{Plánované akce}
Čtvrtek 21. září – 23. ročník \textbf{Běhu strmého do zámeckého vrchu} od 16:30 do 18:30. Běží se 199 metrů s~převýšením 29 metrů v~libeňském parku. Přijďte všichni, ať překonáme rekord v~počtu účastníků (200), a trénujte, aby padl i rekord trati 36,25\,s nebo rekord vaší kategorie. Vyhlášení výsledků v~19:00 v~místě startu. 

Součástí programu bude za příznivých podmínek i promítání skladeb pro slet v~roce 2024 a Sletová štafeta.

Cvičitelé se po běhu strmém sejdou na pracovní schůzce, kde si rozdělí pořadatelství akcí a další úkoly.

Ve čtvrtek 5. října proběhne v~sokolovně a na korábském hřbitově \textbf{pietní akt k~Památnému dni sokolstva}. Proslovy, písně, květiny k~pamětním deskám v~sokolovně i po Libni, vypuštění lodiček se svíčkami na Rokytce za umučené sokoly za druhé světové války. Zveme veškeré členstvo.

\textbf{Podzimní Výlet Libeňského Sokola} (již 65. v~pořadí) bude v~sobotu 7. října. Opět se připojíme ke Srazu v~přírodě pražských žup. Bude na nás čekat Modrá stuha a různé hříčky a vypuštění lodiček se svíčkami k~Památnému dni sokolstva. Budou i Pamětní lístečky, Písnička a další tradiční součásti setkání. Bližší informace ve zvláštním dopisku koncem září. S~sebou jídlo a pití na celý den, odpovídající oblečení, peníze na jízdné a srazový příspěvek. Těšíme se na hojnou účast v~barevné podzimní přírodě.  

Teď se na nové žáky v~Sokole těší a do podzimu pevné zdraví přeje

\signature{Jiří Novák (Jirkan)}{místonáčelník\\tel.: 602 284 198}

\vspace*{24pt}

\post{Zpráva náčelnice}
Je září, začal školní rok a s~ním i další rok sokolské činnosti. Po šesti letech nás čeká rok sletový.
Nácviky, secvičné srazy a oblastní slety pak budou korunovány hromadným vystoupením na
začátku července.

\vspace*{6pt}\noindent
V~našich oddílech budeme nacvičovat tyto skladby:

Rodiče a děti –⁠ Čmeláčci

Předškoláci –⁠ Mravenci

Mladší žákyně – Čarodějky

Starší žactvo – Fitness

Dorostenky – Leporelo

Ženy – V~rytmu srdce

Starší ženy – Babí léto

Senioři a seniorky – Jdi za štěstím

\vspace*{6pt}\noindent
O~možnostech nahlášení účasti na sletu a formě nácviků se informujte u~vedoucích Vašeho oddílu. Vždyť slet často přináší nejen nové zkušenosti a zážitky, ale také často nová přátelství.

Cvičitelská základna ženských složek se blíží počtu 40. Snažíme se však stále školit nové a nadšené cvičitele, neboť i v~našich řadách je značná fluktuace, a to i mezi jednotlivými oddíly. Není to však neláskou k~Sokolu, často je důvodem přestěhování a přechod do jiného Sokola, mateřské či pracovní povinnosti. Naši cvičitelé jsou dobrovolní, ale o~to více nadšení. Přece jen Sokol bere většina z~nás jako „životní styl“ a drží nás zde i přátelství a dobrá parta.

\signature{Alena Krásová (Ája)}{}

\vspace*{18pt}

\post{Zpráva z~oddílu žákyň a dorostenek}
\textbf{Oddíly mladších žákyň a také starších žákyň a dorostenek cvičí shodně v~pondělky a čtvrtky, a to mladší v~čase 17–18 h, starší 18–19 h}.

Náplň hodin se obměňuje v~závislosti na ročním období. Vzhledem k~rekonstrukci šaten se ještě budeme snažit využít hezkých dní k~pobytu venku. Můžete se těšit na venkovní hodiny s~trochou atletické průpravy, her a třeba i výběhů do okolí sokolovny. S~horšícím se počasím pak dostává přednost tělocvična, kde je prostor pro cvičení na nářadí, s~náčiním, ale i průpravu ke sportovním hrám (florbal, házená, volejbal, basketbal, …).

Každá hodina začíná rozcvičkou, obsahuje drobné posílení a protažení. Spíše než o~sportovní výkony nám jde radost z~pohybu a budování dobré party. Ale i ambiciózní jedinci si přijdou na své – na jaře se tradičně účastníme závodů v~sokolské všestrannosti (plavání, atletika, sportovní gymnastika, šplh).

Oba oddíly přijímají nové členy. Těšíme se!

\signature{Alena Krásová (Ája)}{}

\post{Slet 2024}
Všesokolský slet 2024 – možná se to zdá předčasné, ale není – další slet se nezadržitelně blíží. Již skoro před rokem, 26. 11. 2022, byly představeny sletové skladby, které si cvičitelé na videu prohlédli během Silvestra cvičitelů. Už víme, které skladby budeme v~Libni nacvičovat.

1. 10. proběhne finále Sletové štafety. V~Hostivařské sokolovně proběhne doprovodný program naší župy a následně vybraní zástupci jednot odjedou s~jednotovými štafetovými kolíky historickou tramvají do Tyršova domu. Tím oficiálně začnou přípravy na všesokolský slet. A~to také bude start nácviku jednotlivých skladeb. Na naší akademii v~listopadu 2023 uvidíte ukázky některých z~nich. Generálkou budou župní slety během jara 2024. Ten náš bude 25.–26. května 2024 v~Brandýse nad Labem. Vlastní slet se bude konat od 1. července 2024 (to bude Sletový průvod) až do 5.–6. 7. (sletová vystoupení).

V~Libni je hezká tradice účasti na sletech už od roku 1891, kdy cvičilo 21 mužů. V~roce 1920 (dva roky po první světové válce) cvičilo mimo jiné složky na sletě i 320 (vybraných!) libeňských mladších žáků. Po obnovení Sokola v~roce 1990 se konal slet v~roce 1994 – z~Libně cvičilo 111 cvičenců (z~toho 39 žáků a 16 mužů). Ani na sletech v~letech 2000, 2006, 2012 a 2018 nikdy naše účast na sletě neklesla pod sto lidí. Na posledním sletě cvičilo z~Libně 146 členů.

Popřemýšlejte proto prosím o~účasti na sletu a podpořte náš Sokol a naše cvičitele ve sletovém snažení, až se to na podzim tohoto roku dá vše do pohybu. Cvičit nemusí pouze děti, ve skladbách pro dospělé rádi uvítáme i rodiče. Tak se moc těšíme.

\signature{Jiří Novák (Jirkan)}{místonáčelník\\tel.: 602 284 198}

\vspace*{24pt}

\post{Předškolní děti}
Do oddílu předškolních dětí jsou přijímány děti od 4 let. V~hodinách se věnují všestrannému cvičení, naučí se základům atletiky (např. technika hodu) i gymnastiky (např. kotoul vpřed).

V~letošním školním roce se budeme účastnit závodů, kde mohou mladí sportovci zúročit svoje nově nabyté dovednosti.

Velice nás těší, že o~oddíl předškolních dětí je stále velký zájem. Snažíme se vyhovět všem, proto by nám pomohlo, kdyby i někteří rodiče byli ochotni se převléci do sportovního a na sále nám pomohli. Chodit s~dětmi čůrat, zavazovat 
jim tkaničky či vysmrkat je umí z~rodičů každý. :-)

\subpost{Akce, které nás snad čekají}
Dne 21. 9. nebude cvičení, aby se děti mohly zúčastnit \textbf{Běhu strmého do zámeckého vrchu}, který začíná už v~16:30 v~Thomayerových sadech. Více informací naleznete na www.sokol-liben.cz.

\subpost{Slet 2024}
Nácvik na sletovou skladbu bude probíhat od října, a to vždy v~pátek od 16:00 do 17:00. Bude to ve stejný čas, jako budou cvičit rodiče a děti, protože budeme střídat velký a Alšův sál. Tedy v~jeden čas se budou nacvičovat obě skladby, abyste do Sokola nemuseli chodit vícekrát.

V~případě jakýchkoliv dotazů, nejen k~připravovaným akcím, kontaktujte vedoucí cvičení:
 
\signature{Dana Cejpková}{tel.: 606 551 223\\e-mail: cejpkova.dana@seznam.cz}

\vspace*{24pt}

\post{Oddíl rodičů a dětí}
Děkuji, že zvládáte převlékání v~„polních“ podmínkách. Nové šatny za to budou stát. :-)

Dne 21. 9. není cvičení, protože bude Běh strmý, na který jste všichni zváni. Závodit může opravdu každý.

\subpost{Slet 2024}
Nácvik na sletovou skladbu bude probíhat od října, a to vždy v~pátek od 16:00 do 17:00. Bude to ve stejný čas, jako budou cvičit předškolní děti, protože budeme střídat velký a Alšův sál. Tedy v~jeden čas se budou nacvičovat obě skladby, abyste do Sokola nemuseli chodit vícekrát.

Těšíme se na značkách.


\signature{Jana, Dana a Jana}{cvičitelky rodičů a dětí}

\vspace*{24pt}

\post{Ženy – všestrannost}
Léto bylo pro náš oddíl klasicky rozlítané, po červnovém zakončení cvičebního roku se některé z~nás vydaly stavět tábor do Tajanova u~Velhartic, jiné jely do Chrudimi reprezentovat se souborem Nástup sokolskou divadelní tvorbu. V~době prázdnin jsme se různě střídaly na táborech a brigádách v~sokolovně, stihly jsme ale i pár hodin klasického cvičení a dvoudenní splutí Ohře. 

Na nadcházející měsíce máme velké plány: navzdory rekonstrukci šaten pokračujeme v~pravidelném cvičení v~pondělí a ve čtvrtek od 19:00 v~lodích Strnadova sálu. Kromě toho nás čekají i středeční nácviky sletových skladeb „V rytmu srdce“ a \luv{}Leporelo\ruv{}. Máme radost z~velkého zájmu žen o~účast na sletě a rády přivítáme nové tváře i na běžném cvičení.

Pohodový podzim přeje za libeňské Ženy

\signature{Dubina}{}

\vspace*{24pt}

\post{Zuzka}
Jsme před libeňskou synagogou. Tiše postáváme, Zuzka sedí na obrubníku v~obleku Asterixe, a tak jako po filmu běží titulky, doznívá v~nás konec káněcího tábora. Nemluvíme, užíváme si ticho, kterého bylo v~posledních dnech opravdu málo. Jsme vyuzení, utahaní, vyslunění, nevyspalí. Těšíme se na sprchu, postel, ticho, jídlo a snad i jeden den úplného odpočinku. Zároveň máme radost z~toho, že děti měly radost. Máme radost z~toho, že plné nadšení vypráví rodičům svoje zážitky a že se na táboře něco naučily.

Pak si řeknem čau, jdem domu a je konec.

Říkal jsem si v~tu chvíli v~duchu, jak je to divný. Že by měly znít salvy a Asterixe Zuzku na tom obrubníku zasypávat květiny.

Zuzka se tak totiž rozloučila s~dlouholetým vedením oddílu. Svůj volný čas a energii věnovala schůzkám, výpravám a táborům a dělala to opravdu dobře. Dokázala snad vtlouct několika ročníkům Káňat ohleduplnost a lásku k~přírodě, nějakou tu samostatnost, kamarádství, pečlivost, poctivost a nehroutit se z~neúspěchu.

Zuzka nám naštěstí nemizí ze Sokola úplně, bude nacvičovat na blížící se slet a cvičit s~ženami. Tak až ji v~sokolovně potkáte, krom pozdravu se na ni hezky usmějte a v~duchu se pokloňte až k~zemi. Máme v~naší jednotě spoustu inspirativních a skvělých osobností a Zuzka je jednou z~nich.

\signature{Josef Kubišta}{náčelník}

\vspace*{24pt}

\post{Turistický oddíl JILM a Veverky}
Začal krásný nový slunečný školní rok, ptáčci zpívají a nás čeká nová etapa oddílu. Vzhledem k~tomu, že nás stále trápí snížený počet vedoucích (Kondor a Liška jsou v~zahraničí), jsme se rozhodli oficiálně spojit naše oddíly v~jeden. Díky tomu nás je momentálně dost a můžeme hrát víc her pro více lidí. V~programu jsme se snažili skloubit oba oddíly, aby Veverky nechodily na jilmácké schůzky či naopak. V~průběhu roku také chceme rozjet celoroční hru, jež bude začínat na výpravách a schůzkách a vyvrcholí na táboře. Tábor bude společný, na 3 týdny, bližší informace budou.

Klubovnu okupujeme každou středu od 17:00. Na těchto schůzkách hrajeme spoustu her: edukativní, stmelovací či jenom pro zábavu. Program se snažíme mít co nejvíc všestranný, aby si ho užili jak holky i kluci, tak nováčci i staří mazáci. Abychom se mohli těšit nad postupem členů, vytvořili jsme nové oddílové stupně, které v~sobě nesou to nejlepší z~Veverek i Jilmu. Rozdělili jsme se do čtyř družin (tradičně zelená, modrá, žlutá + nová červená). Jednou za čas také chceme uspořádat speciální schůzku, při které půjdeme na výstavu, do bazénu, kina nebo si v~klubovně uspořádáme soutěž ve vaření, signalizaci nebo třeba uzlování.

\vspace*{12pt}\noindent
Akce minulé:
\begin{itemize}[
  itemsep=-3pt,
  leftmargin=2em,
  itemindent=-1em
]
  \item[] 3. 6. jsme se vypravili na jednodenku na Kokořín.
  \item[] 30. 6. –⁠ 2. 7. jsme s~mnoha obětavými pomocníky postavili na louce u~Tajanova tábor od týpek přes kuchyň až k~latríně.
  \item[] 30. 6. –⁠ 22. 7. Jilmáci pluli po vodě i souši za antickými hrdiny (Achillem, Odysseem či Heraklem).
  \item[] 12. 8. –⁠ 24. 8. se tábořiště změnilo na vězeňský tábor, ve kterém měly Veverky za úkol postavit železnici ze Sydney do Darwinu, aby se mohly osvobodit.
  \item[] 25. 8. –⁠ 28. 8. Veverky s~dalšími 16 lidmi spakovali tábor do stodoly nebo odvezli do sokolovny v~dešti i slunci.
\end{itemize}
\vspace*{6pt}
\noindent
Akce budoucí:
\begin{itemize}[
  itemsep=-3pt,
  leftmargin=2em,
  itemindent=-1em
]
  \item[] 22. 9. –⁠ 23. 9. (pá–⁠so) přespání v~sokolovně a podzimní jednodenka
  \item[] 7. 10. –⁠ 8. 10. (so–⁠ne) PVLS + kolovka
  \item[] 11. 11. (so) Vyzvědači
  \item[] 12. 11. (ne) brigáda
  \item[] 16. 12. –⁠ 17. 12. (so–⁠ne) dvoudenní Vánoční výprava
  \item[] 20. 12. (st) Vánoční nadílka
  \item[] 12. 1. (pá) lezení a filmová noc
  \item[] 1. 2. –⁠ 4. 2.  (čt–⁠ne) Zimní táboření/lyže
\end{itemize}

\noindent
Pokud máte jakékoliv dotazy, připomínky nebo nápady, neváhejte se zeptat na našem e-mailu: jilmveverky@sokol-liben.cz

Za společný oddíl
 
\signature{Bára Jeníková}{}

\clearpage

\post{Se Sokolem do divadla}
V~rámci pokračování obnoveného cyklu SE SOKOLEM DO DIVADLA v~současné době nabízíme návštěvu loutkového představení Kašpárek a zbojník, které se bude konat v~Divadle Palace v~pátek 6. října 2023 od 19:00. Termín pro rezervaci vstupenek (jednotná cena 199\,Kč) je v~tuto chvíli stanoven na čtvrtek 14. září 2023. Podle zájmu a disponibility volných vstupenek hodlám následně rezervaci upravovat, nebojte se mě proto kontaktovat na uvedený e-mail i po stanoveném datu.
Představení je vhodné pro všechny kulturychtivé jedince starší 10 let.
Slovy principála našeho sokolského souboru, který již uvedený kus viděl, se jedná o~mimořádně vtipné a povedené představení plné humorných momentů pramenících z~česko-slovenského \luv{}pošťuchování\ruv{} (více informací a detailů Vám zasílali vedoucí oddílů v~samostatném e-mailu; v~případě zájmu rád pošlu znovu).

Zároveň nabízíme pro opožděné zájemce i několik koncertních představení. Jedná se o~výběr koncertů v~Rudolfinu v~rámci vzdělávacích programů určených pro děti. O~detailech jste byli informováni samostatným e-mailem ze strany vedoucích jednotlivých oddílů ještě v~minulém školním roce (na požádání rád přepošlu kompletní informace). Proto zde jen stručný výčet našeho výběru (první 3 jako abonmá nebo samostatné koncerty):


\renewcommand{\arraystretch}{1}
\begin{itemize}
  \setlength\itemsep{-3pt}
  \item Od Carmen k~Tannhäuserovi – út 24. 10. 2023 od 19:30
  \item Italia Bella! – pá 19. 1. 2024 od 19:30
  \item S~Čajkovským nejen u~Labutího jezera – čt 21. 3. 2024 od 19:30
  \item Hej Romale – ne 12. 11. 2023 od 16:30
\end{itemize}

Uzavírka objednávek sice proběhla k~30. dubnu 2023, pokud byste však měli o~jakýkoli koncert zájem, nebojte se mi napsat! Pokusil bych se zajistit v~Rudolfinu dodatečné vstupenky.

Neváhejte a připojte se k~nám v~naší cestě za kulturou. Rádi Vás vezmeme s~sebou! A~pokud byste měli jakýkoli zajímavý tip na jakýkoli kulturní počin, podělte se o~něj a pošlete jej na e-mail zde níže.

\signature{Miloslav Doupal}{e-mail: mila.doupal@sokol-liben.cz}

\clearpage

\post{Sokolská kapka krve}
\subpost{V~1. pololetí roku 2023 darovalo krev 120 sokolů}
Projekt Sokolská kapka krve pokračuje letos 9. ročníkem. Za 1. pololetí roku 2023 své výsledky nahlásilo 25 jednot, kdy celkem 120 dárců absolvovalo 265 odběrů. 

Opět se objevili někteří nováčci a opět je mezi dárci celá řada těch, kteří darovali krev poprvé. 

Průběžné 1. místo obhájil opět Sokol Komárov z~župy Jungmannovy, kde 36 dárců absolvovalo celkem 65 odběrů.

V~Sokole Libeň darovali krev 3 dárci celkem 3x a tento výkon stačil v~průběžném hodnocení na 18.–⁠19. místo. 

Na podporu projektu se i letos konal hromadný odběr v~Tyršově domě a také letos nezůstane jen u~jednoho: podzimní společný odběr bude opět u~příležitosti Památného dne sokolstva v~pátek 6. 10. (podrobnosti s~možností přihlášení zde: \href{https://sokol.eu/akce/sokolska-kapka-krve-hromadny-odber-michnuv-palac-6-10-2023}{Sokolská kapka krve –⁠⁠ hromadný odběr –⁠ Michnův palác 6. 10. 2023}).

Veškeré informace o~projektu a plánovaných hromadných odběrech najdete také na sokolskakapkakrve.cz.

Kdo v~Sokole Libeň darujete krev, pošlete mi prosím počet odběrů za 2. pololetí roku 2023 na e-mail vit.jakoubek@sokol-liben.cz do 31.~1.~2024. Zde také rád zodpovím jakékoliv dotazy. 


\signature{Vít Jakoubek}{zdravotník jednoty}

\vspace*{24pt}

\post{Milostivé léto III}

Oddlužovací akce navazuje na předchozí Milostivé léto I~a II a umožňuje zbavit se výhodně dluhů na sociálním pojištění a na daních. Pokud dlužníci v~období od 1. července do 30. listopadu 2023 uhradí původní dluh na dani a/nebo dlužné pojistné a splní další podmínky akce, penále, úroky a další příslušenství dluhu jim budou odpuštěny. To samé platí i pro místní poplatky, umožní-li to obec, kraj nebo hl. m. Praha.

Možná ve svém okolí máte někoho v~tíživé situaci –⁠ upozorněte jej prosím na tuto akci, jejíž podrobnosti jsou k~nalezení např. na následující adrese:
\href{https://milostiveleto.cz/}{Milostivé léto 2023: Zbavte se dluhů a exekucí –⁠ Milostivé léto (milostiveleto.cz)}

\signature{Vít Jakoubek}

\clearpage

\pagestyle{blank}
\newgeometry{margin=1cm}

\vspace*{96pt}

\pagecolor{sokolred}
\color{white}

\noindent {\fontsize{48pt}{56pt}\tyrs
se sokolským

\vspace*{24pt}

\noindent nazdar!}

\vspace*{\fill}

% \color{black}
\begin{center}
Vydává Tělocvičná jednota Sokol Libeň, Zenklova 37, Praha 8

\vspace*{12pt}

Na přípravě tohoto čísla se spolu s~autory jednotlivých textů podíleli:

grafická úprava – Martin Burian | jazyková úprava – Martina Waclawičová \\ editoři textů – Vít Jakoubek, Jan Přech
\end{center}

\end{document}