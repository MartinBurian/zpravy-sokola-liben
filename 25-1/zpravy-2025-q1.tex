\documentclass[11pt]{article}

\usepackage[a5paper,left=2cm,right=1cm,top=1cm, bottom=1.5cm]{geometry}
\usepackage{array}
\usepackage{makecell}
\usepackage[all]{nowidow}
\usepackage{wrapfig}
\usepackage{enumitem}
\usepackage{multicol}

\usepackage{hyperref}
\hypersetup{
    colorlinks=true,
    urlcolor=sokolblue,
    }

\usepackage[czech]{babel}
\usepackage[utf8]{inputenc} 
\usepackage{ellipsis}

\usepackage{fontspec}
\newfontfamily{\tyrs}{Sokol Tyrs}
\newfontfamily{\fugner}{Sokol Fugner}

% \usepackage{lmodern}
% \usepackage[T1]{fontenc} 
\usepackage{anyfontsize}
\newcommand{\titlesize}{\fontsize{56pt}{67pt}}


\usepackage[dvipsnames]{xcolor}
\definecolor{sokolred}{RGB}{228, 5, 33}
\definecolor{sokoldarkred}{RGB}{200, 0, 30}
\definecolor{sokolblue}{RGB}{45, 46, 135}

\usepackage{tikz}
\usetikzlibrary{calc}

\usepackage{fancyhdr}

\fancypagestyle{standard}{%
    \fancyhf{}
    \fancyhead[LO]{%
        \begin{tikzpicture}[overlay,remember picture]
            \fill [color=sokolred] (current page.north west) rectangle ($ (current page.south west) + (1cm,0cm) $);
            \fill [color=sokolred] ($ (current page.north west) + (1.1cm,0cm) $) rectangle ($ (current page.south west) + (1.2cm,0cm) $);
        \end{tikzpicture}
        }
    % \fancyhead[RE]{%
    %     \begin{tikzpicture}[overlay,remember picture]
    %         \fill [color=orange](current page.north east) rectangle
    %             ($ (current page.south east) + (-1cm,0cm) $);
    %     \end{tikzpicture}
    %     }
    \fancyfoot[C]{%
      \begin{tikzpicture}[overlay,remember picture]
        \fill [color=sokolred] ($ (current page.south east) + (-1.5cm,1.3cm) $) rectangle ($ (current page.south east) + (0cm,0.5cm) $)
         node [pos=0.5,color=white] {\large\tyrs{\thepage}\hspace*{0.5cm}};
      \end{tikzpicture}
    }

    \renewcommand{\headrulewidth}{0pt}
    \renewcommand{\footrulewidth}{0pt}
}



\fancypagestyle{uvodnik}{%
    \fancyhf{}
    \fancyfoot[C]{%
      \begin{tikzpicture}[overlay,remember picture]
        \fill [color=sokolred] ($ (current page.south east) + (-1.5cm,1.3cm) $) rectangle ($ (current page.south east) + (0cm,0.5cm) $)
         node [pos=0.5,color=white] {\large\tyrs{\thepage}\hspace*{0.5cm}};
      \end{tikzpicture}
    }

    \renewcommand{\headrulewidth}{0pt}
    \renewcommand{\footrulewidth}{0pt}
}

\fancypagestyle{blank}{%
    \fancyhf{}
    \fancyfoot[C]{}

    \renewcommand{\headrulewidth}{0pt}
    \renewcommand{\footrulewidth}{0pt}
}


\newcommand{\post}[1]{%
\begin{center}
{\huge \tyrs #1}
\end{center}
}

\newcommand{\subpost}[1]{%
\vspace*{12pt}
\begin{center}
{\Large \tyrs #1}
\end{center}}

\newcommand{\signature}[2]{%
  \begin{flushright}
    \textbf{#1}\\#2
  \end{flushright}
}

\newcommand{\luv}{\clqq\kern-0.07em}
\newcommand{\ruv}{\kern0.07em\crqq\kern0.1em}

\usepackage{csquotes}
\DeclareQuoteAlias{german}{czech}
\MakeOuterQuote{"}

\usepackage[normalem]{ulem}



\babelhyphenation{Ra-kou-ska-Uher-ska}
\begin{document}

%% title
\newgeometry{margin=1cm}
\pagecolor{sokolred}
\color{white}
\pagenumbering{gobble}
\begin{center}

\vspace*{\fill}

{\titlesize \fugner ZPRÁVY}

{\titlesize \tyrs SOKOLA LIBEŇ}

\vspace*{1cm}

{\large ročník LI · číslo 1 · březen 2025}

\vspace*{\fill}
\end{center}

\clearpage
\normalcolor
\nopagecolor
\pagenumbering{arabic}

%% úvodník
\pagestyle{uvodnik}
\newgeometry{margin=1.5cm}


\clearpage

%% úvodník2
\pagestyle{uvodnik}
\newgeometry{margin=1.5cm}

\setlength{\columnsep}{-2.5cm}
\begin{multicols}{2}
  {\fontsize{48pt}{57pt} \fugner \color{sokolred} \noindent Úvodník}

  \columnbreak

  \vspace*{-4pt}

  {\hfill\textbf{Jan Kerhart (Kerhy)}}
  
  {\hfill\textbf{Místonáčelník}}
 \end{multicols}

\vspace*{12pt}

\noindent
Nazdar!

\noindent
Když mi Vítek napsal, zda bych se mohl ujmout úvodníku, pro inspiraci jsem si prošel posledních pět Zpráv a jejich úvody. Zjistil jsem, že každý z~autorů píše o~úplně rozdílných tématech a jedinou spojitostí mezi nimi je Sokol Libeň. O~moc moudřejší jsem z~toho nebyl, a tak jsem se rozhodl to pojmout taky po svém jako všichni předchozí autoři.

Do naší sokolovny chodím již od útlého věku. Postupně jsem si prošel rodiče s~dětmi, kam mě tou dobou vodil dědeček. Protože tam se svými nejmenšími chodily jen maminky, tak příchod dědečka dal vzniknout novému, pro mě nezapomenutelnému pozdravu "Děti, maminky a dědečku, nazdar!" Pak jsem si prošel předškoláky, mladší žáky, Káňata, starší žáky a dorostence. Jak jsem takhle postupoval těmi našimi oddíly všestrannosti, uvědomil jsem si, že mě cvičitelé nevedou jen k~tělesné zdatnosti, ale také jakémusi osobnostnímu rozvoji a utváření si vlastních hodnot. To se mi moc líbilo a začal jsem se více zajímat o~Sokol samotný a také o~to, jak bych se sám mohl stát cvičitelem. 

Proběhl XVI. všesokolský slet a já se rozhodl, že se cvičitelem opravdu stanu. Během těch let jsem si v~Sokole našel spoustu kamarádů a potkal nespočet úžasných a inspirativních lidí. Díky nim mám mnoho nezapomenutelných zážitků, které bych jinde jen těžko získával. Moje sokolské sestry a bratři se pro mě stali druhou rodinou. Za tohle všechno vděčím dědovi, který mě od malička vodil na cvičení a budoval ve mně vztah k~pohybu. Stejně důležitou roli pro mě hrála a hraje ta úžasná parta lidí, kterou u~nás v~Libni máme.

Minulý rok byl plný všemožných aktivit a akcí, jak už je u~nás zvykem. Některé byly větší a některé menší. Přece jen ruchem žijeme. Teď nás čeká v~porovnání s~minulým opravdu klidný rok, takže se nebojte do Sokola na cvičení vodit opravdu i ty nejmenší nebo kohokoliv jiného. Je dost možné, že jako já a mnoho dalších získají úžasnou partu přátel, se kterou je každá akce i jakékoli cvičení nezapomenutelný zážitek.

\clearpage

%% termínka

\pagecolor{sokolred}
\color{white}
\renewcommand{\arraystretch}{1.2}

\newcommand{\boxheight}{11.3cm}

\vspace*{\fill}
\post{TERMÍNOVÁ LISTINA AKCÍ JEDNOTY}
\vspace*{0pt}

\begin{center}
\begin{tikzpicture}
  \draw [ultra thick,color=white](0.3cm,0cm) rectangle (12.3cm,\boxheight);
  \fill [color=white] (0cm,0.3cm) rectangle (12cm,\boxheight + 0.3cm)
  node [pos=.5, color=black] {
    \begin{tabular}{l  p{6.5cm}}
      2. 3. (ne) & jarní brigáda \\
      8. 3. (so) & Dětské a Dospělácké Šibřinky \\
      19. 3. (st) & Valná hromada (volební) \\
      5. 4. (so) & gymnastické závody PD \\
      12. 4. (so) & 68. Jarní Výlet Libeňského Sokola \\
      25.–27. 4. (pá–ne) & župní závody všestrannosti \\
      8. 5. (čt) & Praha Open – gymnastické závody \\
      17. 5. (so) & Open house – Den otevřených dveří \\
      16.–18. 5. (pá–ne) & finále Zálesáckého závodu zdatnosti \\
      29. 5. (čt) & Dětský den a loutkové představení \\
      8. 6. (ne) & atletické závody R+D a PD \\

      & \\
      27. 6. – 19. 7. & tábor oddílu Veverky a JILM   \\
      5.–12. 7. & tábor oddílu Venkovních sportů – Janské Lázně \\
      12.–19. 7. & pobyt rodičů a dětí – Janské Lázně \\
      19.–26. 7. & tábor bývalých členů JILMu \\
      26. 7. – 9. 8. & tábor oddílu Káňata \\
    \end{tabular}
  };
  %  node [color=black] {FOOBAR};
\end{tikzpicture}

\renewcommand{\arraystretch}{1}

\vspace*{12pt}
Podrobné informace k~akcím budou zveřejňovány na našich internetových stránkách a distribuovány dalšími obvyklými kanály (cvičitelé, vývěsky, e-mail)

\end{center}
\vspace*{\fill}

\clearpage
\nopagecolor
\normalcolor

%% normální obsah
\restoregeometry
\pagestyle{standard}

\post{Zpráva místonáčelníka}
Nejprve stručný výčet již proběhlých akcí

Po podzimním hektickém mumraji okolo oslav 140. výročí založení jednoty přinesl prosinec relativní klid a naše tradiční akce: 

Mikuláš s~8 čerty a 10 anděly se v~sokolovně objevil přesně ve čtvrtek 5. prosince. Od 16 hodin se byl podívat na cvičení rodičů a dětí a předškoláků a od 17 hodin na cvičení mladšího žactva. U~žactva se po rozcvičce a hře setmělo a do sálu osvětleného svíčkami vnikli čerti, aby si odnesli zlobivá dítka. Nikoho si ale neodnesli (zlobivci se prý polepší). Pak už si Mikuláš v~klidu poslechl spoustu básniček a slyšel také písničky za doprovodu akordeonu. Celkem andělé rozdali 140 balíčků.

Rychlý šplhoun v~Sokole Kobylisy proběhl za účasti 9 Libeňáků v~neděli 8. prosince. Vítězství si odnesla štafeta mladších žáků, 2. místo získal M. Trnka v~mladších žácích, 3. místo pak muži J. Kubišta na laně a L. Šimek na tyči. 

Vánoční nadílka turistického oddílu JILM a Veverky byla ve středu 18. prosince v~klubovně. Koledy, vánoční povídání, cukroví, vánoční povídka, dárky, stromek. Bylo 10 Jilmáků, 6 Veverek a 9 bývalých členů a hostů.

Vánoční loutkové představení Vánoční sokol v~provedení našeho loutkového divadla Nástup! se za účasti 110 dětí a 55 dospělých konalo ve čtvrtek 19. prosince.

Silvestr cvičitelů se konal v~pátek 3. ledna 2025. Občerstvení z~vlastních zdrojů, povídání ve sborovně, půlnoční přípitek. 19 cvičitelů, 9~cvičitelek, 15 hostů.

\subpost{Informace z~cvičení}
Ml. žáci (2015–2018) cvičí v~út a čt v~17–18, st. žáci a dorostenci (2007–2014) v~út a čt v~18–19. Pro cvičitele a dorostence možnost úterního cvičení mužů a čtvrteční košíkové (oboje 19–20). 
O~kluky se stará náš mužský cvičitelský sbor v~tomto složení:
Cvičitelé: J. Novák (55), T. Novák (53), J. Kudroň (37), J. Přibyl (34), J. Kubišta (33), J. Kerhart (23), P. Boháč (22), J. Skokan (21), T. Kléger (20) a naši mladí pomahatelé: V. Novák, V. Blahunek, J. Pikálek a A. Basseville. 

Pevně věříme, že kluci díky nám (protože cvičení je většinou baví) i díky Vám přivedou další nové kamarády – rádi uvítáme další zájemce o~cvičení (bereme mladší i starší žáky). Děkujeme a těšíme se. 

V~loňském cvičebním roce jsme měli zapsáno 136 cvičenců (67 mladších žáků, 39 starších žáků, 13 dorostenců a 17 cvičitelů) a průměrná docházka na hodinu byla 59,835 (o~více než 7 lepší než v~roce předchozím) při měsíčním průměru zapsaných 106,9. Letos je zatím zapsáno 127 cvičenců (55 mladších žáků, 49 starších žáků, 13 dorostenců a 10 cvičitelů) a průměrná docházka na hodinu za září–leden byla 60,56. Bohužel půjdeme s~průměrem kousek dolů, protože po prosinci a lednu, kdy byla účast kvůli nemocím nižší, přišel únor, kdy na kluky dolehly nemoci ještě výrazněji a na hodinu mladších žáků dorazilo třeba i jen pouhých 10 kluků oproti 30 a více za běžného stavu. Tak ať se rychle uzdraví a vrátí se nám do cvičení.

Od prosince běží i nová zimní soutěž Hod, kop, střela, která má tři disciplíny. Cílem je trefit co nejvíckrát po sobě bez přerušení malou florbalovou branku. Poprvé hodem malým míčem, podruhé kopem malým míčem a potřetí florbalovým míčkem střelou florbalovou hokejkou. Obtížnost je určena předepsanou vzdáleností pro jednotlivé věkové kategorie. 

Od ledna pak začal pozvolný nácvik na únorové nominační závody v~gymnastice. 

V~lednu jsme též vyhlásili výsledky soutěže Letní disciplíny. Kluci za své atletické snažení dostali diplomy a sladkou odměnu a ti nejlepší si mohli vybrat i z~drobných věcných cen. Výsledky jsou vyvěšeny na nástěnce vedle vrátnice. Mezi 76 mladšími žáky ve svých ročnících zvítězili: M. Mikyška, A. Příhoda, D. Hájek, M. Cakl a L. Bednář a dobré výkony ještě předvedli: J. Černý, A. Musil, F. Gaberle, M. Pošta, F. Vinkler, R. Marin, O. Boušek, O. Nekvapil, V. Zelený, J. Voráč a O. Ševčenko. Mezi 39 staršími žáky v~ročnících zvítězil: M. Trnka, J.~Smutný, J. Šefrna a H. Hőschl a jmenování si dále zaslouží: Š. Novák, O. Doupal, J. Doupal a S. Budai. Mezi 15 dorostenci a cvičiteli byli nejlepší: V. Blahunek, T. Kléger, J. Kubišta, J. Pikálek, J. Kerhart, V.~Novák a P. Boháč.

{\sloppy
Trvá také celoroční soutěž O~nejvěrnější docházku s~pravidelným každoměsíčním vyhlašováním a rozdáváním diplomků za 100\% docházku.}

V~týdnu před jarními prázdninami proběhly 11. a 13. února Nominační závody žáků v~gymnastice a šplhu, po kterých (i s~přihlédnutím k~výkonům v~naší soutěži Letní disciplíny) vybereme ty nejšikovnější kluky pro další nácvik na závody všestrannosti, které se budou konat 25.–27. 4. 2025 a kde se bude závodit v~plavání, gymnastice (přeskok, prostná, hrazda, kruhy a starší i bradla), šplhu a atletice (sprint, skok daleký, delší běh a hod krikeťákem). S~ohledem na nemocnost a malou účast bude proveden výběr až první březnové cvičení po docvičení chybějících kluků.

Memoriál Jana Vorla se uskutečnil v~sobotu 15. února. Nejprve jsme se sešli na hřbitově a Hnízdovi zapálili svíčky a vzpomněli na něj. Po přesunu do sokolovny došlo na vlastní závody v~gymnastickém pětiboji (prostná, kruhy, hrazda, bradla, přeskok), kterých se účastnilo pod dozorem 7 rozhodčích 17 borců, které sledovalo 12 diváků. Před vyhlášením výsledků se ještě uskutečnil závod ve šplhu na laně s~nově zakoupenou automatickou časomírou. 

\subpost{A~co nás čeká ve cvičení v~nejbližších dnech a měsících?}
V~neděli 2. března (od 9 do 14 hodin) bude tradiční jarní brigáda. Zveme cvičitelstvo, starší žactvo, dorost a Jilmáky s~Veverkami. Pokud se zapojí muži, ženy nebo rodiče, budeme rádi. Práce je uvnitř sokolovny od sklepa po půdu a venku okolo celé sokolovny, dokonce i za plotem. Dvě největší práce budou venku. Na základě požadavku památkářů musíme odstranit betonový obklad z~nové podzemní místnosti – příklepové vrtačky s~majzlíky s~sebou. A~pak se bude muset urovnat dovezená hlína na místě původní plechové garáže, abychom mohli následně zasít trávu.

V~sobotu 8. března se budou konat Šibřinky. Nejprve odpoledne dětské, večer pak taneční zábava s~živou kapelou pro dospělé. Bližší informace již najdete na našich stránkách a také Vám došly přes EOS.

Ve středu 19. března od 18 hodin proběhne Valná hromada. Pokud chcete být delegátem za váš oddíl (je vám více než 18 let a jste členem naší jednoty), informujte svého cvičitele a nechte se zvolit zástupcem vašeho oddílu. Za každých 30 členů oddílu jeden delegát. Letošní Valná hromada bude volební – bude se volit nový výbor na další 3 roky. Pokud splňujete podmínku delegáta a chtěli byste pomoci naší jednotě prací ve výboru, přihlaste se mi. Probereme podrobnosti a umístíme Vás na kandidátku.

68. Jarní Výlet Libeňského Sokola (JVLS) je naplánován na sobotu 12. dubna. Čeká nás tradiční program se Zálesáckým závodem zdatnosti (ZZZ) pro děti, hry, koupání, oheň k~obědu, Jazykohrátky, Pamatovačka, písničky. Připojte se k~nám. Těšíme se. Podrobné informace cca 14 dní předem dostanete přes EOS.

Dále nás čekají Závody všestrannosti pro vybrané kluky (25.–27.~4.), gymnastické závody Praha Open pro postupující žáky z~všestrannosti a pro muže (8. 5.), Finále ZZZ – pokud některá trojka postoupí během JVLS (16.–18. 5.), otevření sokolovny pro veřejnost v~sobotu 17.~5. v~rámci akce Open house a Dětský den (29. 5).

\subpost{Turistické oddíly}
Všem žákům nabízíme možnost účasti na výletech, které pořádají naše turistické oddíly. Káňata zvou mladší žáky, Jilmáci žáky starší. Mladší děvčata mohou chodit do Káňat, starší do Veverek. Všechny oddíly též samozřejmě nabízejí i členství – což znamená středeční (JILM + Veverky) či čtvrteční (Káňata) schůzky v~klubovně a výpravy, které se konají cca 1x měsíčně. Vyvrcholením celoroční činnosti je pak letní tábor. V~roce 2025 to bude nové místo nedaleko Trhových Svin v~jižních Čechách – na Keblanském potoce mezi obcemi Něchov a Keblany. Informace u~vedoucích oddílů – viz informace u~oddílů na našich stránkách. 

\subpost{Cvičení mužů a dalších oddílů}
Líbí se vám cvičení mužů na akademiích? Klidně se můžete stát členem oddílu – berou další zájemce (muže 16–50). Cvičí se v~úterý 19–20. 

Výborně funguje před několika lety založený oddíl šplhu (osmimetrové lano ze sedu bez přírazu). Řada jeho členů se účastní i mistrovství republiky v~tomto sportu. Též berou další zájemce od staršího žactva až po muže a šplhají i děvčata a ženy. 

Další větví v~mnohotvárné činnosti mužů je Přetah lanem – i v~tomto sportu se závodí a konají se i mezinárodní soutěže. Pro trénink si za sokolovnou vystavěli cvičnou tahací věž.

\subpost{Organizační informace}
O~jarních prázdninách (18. a 20. února) žáci cvičí. O~velikonočních prázdninách ve čtvrtek 17. 4. žáci také cvičí, naopak ve čtvrtky o~státních svátcích 1. a 8. května je sokolovna uzavřena. 
Cvičení venku s~atletickými disciplínami začne podle počasí, nejdříve však v~úterý 29.~dubna.

Děkujeme za používání matričního systému EOS. Před pár dny byly vypsány příspěvky na první pololetí roku 2025 a už máme na účtu skoro polovinu částky – děkujeme. Platby příspěvků se nám tak loňský podzim i teď na 1. pololetí roku 2025 sešly během několika málo dní. 
Zároveň vás tímto kanálem dokážeme rychle a v~případě potřeby i cíleně informovat o~našich akcích. A~navíc, pokud vy potřebujete něco změnit, máte tak šanci to učinit sami (změna kontaktních údajů, zdravotního stavu apod.), případně lze změnit i oddíl či četnost hodin, a to tak, že se přihlásíte do systému, dáte „Nový požadavek na management“ a napíšete, co potřebujete. Kancelář to učiní a dá vám vědět, že se tak stalo. 

V~matrice – 1. patro sokolovny (čtvrtek 15:45–18:45) lze pak zakoupit bílé tričko se znakem pro žáky.

Na akademii byl představen nový vzhled i uspořádání našeho webu. V~tuto chvíli se do něj nahrávají a upravují data. Ne vše je ještě doladěno. Proto je tam zatím i odklik na starý web. Zá pár týdnů už by měly být naše nové stránky plně funkční. Dělali ho naši mladí cvičitelé, tak jim prosím odpusťme počáteční neduhy. 

Výsledky závodů, fota, dopisy a další informace najdete jednak na vývěskách v~sokolovně a na ní a také na www.sokol-liben.cz. Informace o~důležitých akcích pak putují i přes přihlašovací systém EOS.

Program máme opravdu bohatý, mimo vlastní cvičení dáváme mnoho nabídek na akce s~různým zaměřením. Stačí si jen vybrat a připojit se k~nám. Těšíme se na setkání.

Příjemné jaro všem přeje 

\signature{Jiří Novák (Jirkan)}{místonáčelník\\tel.: 602 284 198}

\vspace*{24pt}

\post{Informace od starosty}

Od posledních Zpráv jsme opět kousek pokročili v~údržbě sokolovny. Co zůstává stabilní, je počet členů, který je stále okolo 800.

\subpost{Finanční záležitosti}
Provedli jsme vyúčtování grantů za rok 2024 a požádali o~další granty pro rok 2025.
Finanční situace se nám již stabilizovala. Za rok 2024 se hlavně díky grantům (1,75 milionu) a platbám od nájemců nebytových prostor (skoro 2 miliony) a sálů (skoro 0,75 milionu) a také platbám od členů za příspěvky (skoro 2 miliony) podařilo ušetřit na další opravy sokolovny přibližně 1 milion Kč. 
Od ledna 2025 došlo k~dalšímu zvýšení minimální mzdy našich zaměstnanců skoro o~10\%, což v~ročním souhrnu znamená cca 200 000 Kč. Za kompletní opravu místností staré matriky (podlaha, světla, vodovod, výmalba, repase a nové dveře, propojení s~nájemcem ABA) zaplatíme skoro 0,5 milionu. Ale zase budeme moci od dubna pronajmout a budou tak obsazeny všechny nebytové prostory v~sokolovně.
Jako jednota, která se zúčastnila sletu, jsme přes ČOS dostali od sponzora Penny 25 000 Kč. Dalším přilepšením by měla být dvoje jarní a letní filmařská natáčení v~sokolovně.

\subpost{Práce kolem sokolovny}
Během roku nás čeká i práce na nové podzemní místnosti – dle požadavku památkářů musíme strhnout betonový obklad a nahradit ho omítkou. Chybu projektu jsme reklamovali u~architektů, ti se obrátili na svoji pojišťovnu a skoro po roce nám přišlo pojistné plnění za zbytečně vykonané práce ve výši 340 000 Kč. Abychom ušetřili, pokusíme se obklad sejmout svépomocí na jarní brigádě. Novou omítku pak zadáme firmě: odhaduji to tak na 100–150 tisíc. Pak už snad konečně bude následovat kolaudace.
Na podzimní brigádě byla stržena plechová garáž, rozbit a odvezen beton a následně zdarma dovezena hlína díky naší cvičitelce Daně Cejpkové. Na jarní brigádě plochu připravíme na osetí trávou. Pak ještě zhotovíme ohniště, kruh na vrh koulí, ohraničíme dráhu pro tahače lanem, kteří si u~opěrné zdi zbudovali tréninkovou tahací věž, a zase bude kousek plochy kolem sokolovny upraven a zkrášlen.
Materiál z~garáže bude na jaře použit na našem novém tábořišti pro zbudování boudy na uskladnění podsad a dalšího táborového vybavení.

\clearpage
\subpost{Práce v~sokolovně}
Již jsou připraveny držáky na lavice do prostoru vedle schůdku do sociálek a každým dnem mají dorazit fošny na lavice. Třeba dojde na montáž už během jarní brigády.
Před pár dny byla opravena kapající trubka v~dámských šatnách. Brzy bude připevněn zpět i krycí plech vzduchotechniky.
Před mrazy jsme zakryli výdechy katakomb na dvoře, aby nedošlo k~zamrznutí přívodu vody do sokolovny, který katakombami vede. 

Během ledna jsem vzorně uklidil dílnu a prostor s~uskladněnými věcmi pod ní, kde byl chaos a nepořádek po stavbě šaten a montáži vzduchotechniky. Teď je vše přehledně umístěno.

\subpost{Připomínky k~provozu v~šatnách}
V~souvislosti s~šatnami ještě jedna prosba ohledně botníků. Pokud si nechci boty zamykat, použiju poličku, pokud ano, použiju malý zamykací šuplík a zamknu ho. Větší šuplíky jsou určené pro dětské oddíly, kam se vejde více párů bot. Neobsazujte nám je vaším jedním párem – děkujeme. Prázdné šuplíky nechávejte odemčené a klíčky vracejte na háček. 
Hlavním přínosem čistého provozu v~šatnách by měla být i větší čistota v~sálech. Zásada, na které budeme stále striktně trvat, je, že žádné venkovní boty nepřekročí práh šaten, a to ani v~ruce!

POZOR – v~rámci drobné reklamace nám bude doručeno cca 30 zámečků od skříněk a botníků. Pokud někdo máte nefunkční či rozpadlý zámek u~skříňky, dejte to prosím vědět do kanceláře nebo starostovi, případně nechte zprávu alespoň ve vrátnici (číslo skříňky, jméno). Následně vás budeme kontaktovat a provedeme výměnu zámku a předání nového klíče – děkuji za spolupráci.
 
\subpost{Připomínky k~provozu v~sokolovně}
Dále prosím všechny cvičence i návštěvníky sokolovny, aby nemanipulovali s~termostaty na radiátorech, a pokud se někde otevře okno, aby se opět zavřelo, a to na obě kličky. A~nakonec prosím i o~zavírání obou dveří do šaten (jednak se tam nedostanou zloději a zároveň se tam nedostane studený vzduch z~haly či zadního schodiště) – děkuji.

\subpost{Plány do blízké i vzdálenější budoucnosti}
Momentálně upravujeme pojistnou smlouvu na sokolovnu, aby bylo vše v~pořádku. 

Stále probíhají každodenní drobné opravy v~sokolovně – máme 800 členů, další stovky cvičenců přicházejí ze stran nájemců. V~tomto čilém provozu se prostě věci opotřebovávají více než třeba doma, a tak je potřeba neustále něco opravovat. V~brzké době budeme muset opravit nízkou zídku z~kyklopského zdiva pod plotem hřiště gymnázia, kde vypadává spárovací omítka i celé kameny. Po podzimní opravě střech dojde i na opravu stropu v~kanceláři ve 2. patře, kde zatékalo z~terasy, a také na opravu stropu a stěny v~nářaďovně Srncova sálu, kde také zateklo. Stále je co dělat. Do roku 2031 bychom také potřebovali dokončit rekonstrukci dvora (předláždění, nová vrata a vrátka, nový plot), protože nám bude končit platnost stavebního povolení.
A~co pak? Pak se vrhneme na vnitřek sokolovny – hlavně výmalbu.


\signature{Jiří Novák (Jirkan)}{starosta\\tel.: 602 284 198}

\vspace*{24pt}

\post{JILM a Veverky}
Drazí čtenáři, i v~letošním roce pokračuje činnost libeňských turistických oddílů. Pro mladší žactvo (6–10 let) je tu oddíl Káňata, který se schází v~klubovně libeňské sokolovny každý čtvrtek. Pro starší žáky a žákyně (10–15 let) jsou tu oddíly JILM a Veverky, které nyní již druhým rokem fungují společně, a to ve středu od pěti do sedmi. Na podzim jsme podnikli spoustu výprav a dalších akcí. Vedle Podzimního Výletu Libeňského Sokola, kterého se také účastnili mnozí z~vás, jsme vyrazili na kola do okolí Prahy, vydali se do sněhem pokryté lednové krajiny a před pár týdny dokonce na prodloužený víkend na lyže do Krkonoš.

Ve druhém pololetí nás čekají další tradiční akce, jako je Jarní Výlet Libeňského Sokola spojený s~oblastním kolem Zálesáckého závodu zdatnosti, ve kterém pravidelně obsazujeme postupové pozice na celorepublikový přebor nebo další soutěže (Mistr uzlování a Mistr signalizace). Během jara nás také čeká spousta práce, jelikož bychom chtěli v~létě poprvé tábořit na novém tábořišti u~Českých Budějovic a louku je potřeba na celoprázdninový provoz trochu připravit.

Všechny naše oddíly mají stále nějaký prostor pro příjem nových členů, a to i na letní tábory. Pokud byste měli chuť podílet se na naší činnosti a zájem o~pomoc s~vedením oddílu Káňat, budeme rádi za každou přiloženou ruku. V~případě jakýchkoli dotazů stran turistických oddílů se neváhejte ozvat e-mailem nebo telefonicky.

Pro informaci přidávám také seznam akcí T. O. JILM a Veverky na další pololetí.

\begin{itemize}[
  itemsep=-3pt,
  leftmargin=1em,
  itemindent=-1em
]
  \item[] 1. 3. (so) Jarní jednodenka
  \item[] 2. 3. (ne) Brigáda u~sokolovny, 9–14 hod
  \item[] 8. 3. (so) Šibřinky – tradiční sokolský maškarní ples, zváni jsou rodiče i starší členové. Mladší členové mohou pomoci jako organizátoři odpoledních Dětských šibřinek.
  \item[] 26. 3. (st) Mistr signalizace na schůzce
  \item[] 9. 4. (st) Mistr uzlování na schůzce
  \item[] 12.–13. 4. (so–ne) JVLS a semifinále ZZZ – Sraz pražského třížupí v~přírodě, proběhne též pražské kolo Zálesáckého závodu zdatnosti. Na sobotu jsou zváni i rodiče, v~neděli navážeme vlastním programem.
  \item[] 16.–18. 5. (pá–ne) Finále ZZZ v~Úpici – pro trojky i náhradníky, kteří postoupí za župu 
  \item[] 30. 5. – 1. 6. (pá–ne) Brigáda na novém tábořišti v~Keblanech s~účastí členstva 
  \item[] 20. 6. – 22. 6. (pá–ne) JaxiTaxi: Pracovka a stěhování – příprava táborové louky, stěhování na nové tábořiště, povinné pro účastníky tábora
  \item[] 8.–10. 8. (pá–ne) Bourání tábora – pro členstvo nepovinné, budeme rádi za pomoc
\end{itemize}

\clearpage

\noindent Tábory:
\begin{itemize}[
  itemsep=-3pt,
  leftmargin=1em,
  itemindent=-1em
]
  \item[] Veverky a JILM    27. 6. – 19. 7. 2025
  \item[] Bývalí členové 19. 7. – 26. 7. 2025
  \item[] Káňata          26. 7. – 9. 8. 2025
\end{itemize}

\noindent
Za turistické oddíly

\signature{Dan Unzeitig (Kondor)}{dan.unzeitig@sokol-liben.cz, 72036708}

\vspace*{24pt}

\post{Prosba všemu členstvu}

Pokud máte jakékoliv sokolské předměty či dokumenty, které byste vyhazovali nebo pro ně nemáte využití, věnujte je do sokolského archivu. Uložíme knížky, cvičební úbory, sokolská trička, fotky tištěné i elektronické, časopisy, diplomy, plakáty a pozvánky, pamětní předměty jako medaile a odznáčky, \ldots{}
Předměty k~věnování můžete nechat ve vrátnici nebo předat osobně vzdělavatelkám (Anka chodí na žákyně a ženy, Bára na turisťáky a žákyně), případně napsat e-mail Ance a domluvit jiné předání.
Děkujeme všem, kteří přispějí k~uchování obrazu o~naší době pro budoucí generace!

\signature{Anka Holanová a Bára Jeníková}{vdělavatelky \\ anna.holanova@sokol-liben.cz}

\vspace*{24pt}

\post{Oddíl rodičů a dětí}
Na cvičení je stále velká účast, a tak určitě všechny potěší, že od čtvrtka 24. 4. bude probíhat cvičení jen na velkém sále, protože předškolní děti budou cvičit venku. Pokud však bude odpoledne pršet a předškolní děti nebudou moci cvičit venku, budou se rodiče a děti automaticky přesouvat do Srncova sálu (ten se zelenou podlahou).

Prosím, začněte přemýšlet, jak budete chtít docházet na cvičení od září 2025. Už nyní je dlouhý seznam zájemců o~cvičení ve školním roce 2025/26. Dotazník týkající se Vašeho zájmu o~cvičení Vám přijde na konci dubna do systému EOS.

\noindent Dny a časy hodin se ve školním roce 2025/26 nebudou měnit:

pondělí 16:00–17:00 

úterý 9:50–10:50 

čtvrtek 16:00–17:00 

Stále hledáme nové cvičitele pro oddíl rodičů a dětí. Pokud Vás cvičení baví a chtěli byste se mu dále věnovat, i když Vaše děti už z~tohoto oddílu odrostou, stačí jen napsat nebo se zeptat na hodině cvičitelky. Všechna potřebná školení za Vás hradí Sokol Libeň. :-)

\signature{Dana Cejpková}{mobil: 606 551 223\\e-mail: dana.cejpkova@sokol-liben.cz}

\vspace*{24pt}

\post{Předškolní děti}
Ve druhém pololetí bude výuka v~hodinách předškolních dětí zaměřena na základy gymnastiky a atletiky. Vše souvisí s~plánovanými závody.

Dne 5. 4. čekají vybrané děti závody v~gymnastice v~TJ Sokol Vršovice. Pravidelná příprava na závody probíhá v~pátek od 16:00 do 17:00. Držte nám palce, ať v~silné konkurenci obstojíme. 

Od 24. 4. budeme chodit cvičit s~dětmi ven na doskočiště, abychom důkladně natrénovali skok do dálky a běh v~dráze, protože nás čekají v~červnu závody v~atletice. Děti budou přicházet do svých šaten, z~kterých budeme chodit na venkovní sportoviště, které je hned vedle sokolovny.

Stále máme na cvičení hodně dětí a zájem o~místa na školní rok 2025/26 je už nyní velký a přesahuje kapacitu oddílu. Další děti nemůžeme přijmout, protože nemáme dostatek cvičitelů, kteří by se mohli dětem věnovat. Pokud tedy máte zájem se zapojit více do chodu libeňského Sokola v~roli pomahatele či cvičitele, stačí jen kontaktovat vedoucí cvičení a říci si, jaké jsou Vaše představy a časové možnosti, jak se do činnosti organizace zapojit.

\noindent Cvičení ve školním roce 2025/26 bude probíhat ve stejné dny a časy:

pondělí 16:00–17:00

čtvrtek 16:00–17:00

Pokud uvažujete o~přechodu dítěte do mladšího žactva, bude mít možnost si během června docházku do tohoto oddílu nanečisto vyzkoušet, aby mělo reálnou představu, zda od září do tohoto oddílu chodit chce, nebo nechce. V~září už bývá problém pro dítě najít náhradní pohybovou aktivitu, protože kroužky/oddíly mají již naplněnou kapacitu.

\signature{Dana Cejpková}{mobil: 606 551 223\\e-mail: dana.cejpkova@sokol-liben.cz}

\vspace*{24pt}

\post{Se Sokolem do divadla}
Kulturní sezónu roku 2025 jsme zahájili 23. ledna komedií Kachna na pomerančích v~podání amatérského souboru Divadlo bez dozoru. Vstupenky byly bohužel rozebrány dřív, než mohla odejít oficiální pozvánka přes EOS všem členům Sokola Libeň. Po domluvě s~principálkou „ukořistila“ naše jednota alespoň 9 balkónových vstupenek, díky nimž jsme se bavili stejně dobře jako diváci v~přízemí.

Skvrnou na povedeném kulturním počinu tak zůstává jen nedostatečný počet vstupenek, nicméně věřím, že příště se podaří zajistit u~divadelního souboru větší počet míst.

Věřím a doufám, že další pozvánka na sebe nenechá dlouho čekat. Než však natrefím na konkrétní vhodné představení, dovolím si sdílet sokolské kulturní tipy v~rámci projektu „Do Sokola za kulturou“, které by Vás mohly zaujmout:

\begin{itemize}[
  itemsep=-3pt,
  leftmargin=1em,
  itemindent=-1em
]
  \item[] Michnův palác: 4. 3. 2025 od 18,00 h – Masopust v~tanci a zpěvu – Vycpálkovci a Rozmarýn
  \item[] Michnův palác: 12. 3. 2025 od 19,00 h – Divoké předjaří v~Michnově paláci – komponovaný pořad s~romskými písněmi – soubor RAŇIJA
  \item[] Michnův palác: 25. 3. 2025 od 19,00 h – Koncert swingové formace Modrá synkopa
  \item[] Sokol Libeň: 31. 3. 2025 od 17,00 h – Klub seniorů Libeň – Přednáška Po stopách zaniklé Karwinné – Šikmý kostel
  \item[] Michnův palác: 6. 4. 2025 od 17,00 h – Komponovaný hudebně-taneční pořad – Sokolské tanečnice z~Dobříše
  \item[] Michnův palác: 10. 6. 2025 od 19,00 h – Koncert swingové formace Modrá synkopa
  \item[] Michnův palác: 11. 6. 2025 od 19,00 h – Anička Jagošová – cimbálovka a Dětský pěvecký sbor Dany Ungrové
\end{itemize}

Pozn: Uvedené sokolské tipy jen přeposílám a upozorňuji, že ve výjimečných případech se může stát, že dojde ke změně data nebo místa konání. Bližší informace naleznete zde: \href{https://prosokoly.sokol.eu/kultura-v-tyrsove-dome}{https://prosokoly.sokol.eu/kultura-v-tyrsove-dome}. Vstupné, většinou symbolické, je hrazeno těsně před vstupem na akci.

A~pokud byste někdo měl jiný zajímavý tip, klidně jej nasdílejte na níže uvedený e-mail.

\signature{Miloslav Doupal}{e-mail: mila.doupal@sokol-liben.cz}
\vspace*{24pt}

\post{Přednáška v~Žižkovském divadle Járy Cimrmana}
\textit{Přednáška v~Žižkovském divadle Járy Cimrmana, jak ji proslovil na slavnostním představení hry České nebe u~příležitosti 140. výročí založení Sokola Libeň dne 24. 11. 2024 náčelník Josef Kubišta}

\medskip

\noindent Drahé sestry, mí milí bratři,

\noindent při přípravě letošního Almanachu ke 140. výročí Sokola Libeň jsme se snažili podat co nejpřesnější informace, a tak jsme nahlédli do více dokumentů našeho archivu. Jedním ze skvostů našeho archivu je stavební deník libeňské sokolovny.

Každý z~nás dobře zná a ví, že stavba naší libeňské sokolovny trvala úžasných 10 měsíců mezi lety 1909 a 1910. A~každý z~nás se sám sebe ptá, jak je možné odkopat takový kus terénu, vystavět byť jen základy, natož celou sokolovnu v~tak krátkém čase, bez využití jakékoli těžké techniky. Stavební deník sice tvrdí, že stavba byla zadána stavební firmě Matěj Blecha z~Karlína, nicméně z~deníku je patrné, že v~prvním půlroce bylo dokončeno 95\% všech prací. A~podle, uznávám nepřímých, důkazů vedl první půlrok stavební práce někdo úplně jiný než Matěj Blecha.

Cituji ze stavebního deníku:

Ten – tady je jméno zaškrtáno – pustil se do ostré hádky s~bratrem starostou Filipem. Ten rozezlen, jak jsme ho ještě nezažili, zvolal: „Nech toho, Jarouši, nebo ti ukáži, kterak jsem dobrý v~rohování!“ A~když se schylovalo k~nejhoršímu, odjel ten nevychovaný stavbyvedoucí na svém fiakru, ze kterého ještě nebyl složen náklad čtyř kop hřebíků a čtyř tuctů prken. Práskl opratěmi a již jsme ho neviděli.

Konec citace

Zázračné tempo stavby a zmínka o~jakémsi Jaroušovi, který nám zcizil prkna a hřebíky, nás nenechala chladnými. A~tak jsme vznesli dotaz na předního českého cimrmanologa pana profesora Miloně Čepelku.

Cituji:

Vážený pane profesore, posíláme vám v~příloze kopii stavebního deníku s~dotazem, jestli se na pozici stavbyvedoucího libeňské sokolovny nemohl náhodou podílet Jára Cimrman. Není možné, že Jára Cimrman vedl stavbu až do doby, než se na nás Sokoly nasupil, tedy do března 1910, kdy i založil tělocvičný spolek SUP? Není možné, že Blecha jen parazitoval na Cimrmanově nedokončené práci?

Konec citace

Tak na tento dopis nám pan profesor neodpověděl.

Nepřímé důkazy se tentokrát potkaly s~naprosto precizním záznamem počtů a statistik našich libeňských starostů.

Protože jsme s~bratrem starostou dnes oba poprvé v~divadle Járy Cimrmana, nedalo nám to a divadlo jsme si opravdu důkladně prohlédli. Všechno si přepočítali. A~já myslím, že je to nad slunce jasné.

Je mi neskutečnou ctí přivítat vás zde, na prknech Sokola Libeň, v~Žižkovském divadle Járy Cimrmana!

\vspace*{24pt}

\post{Sokolská kapka krve}
Výsledky za 10. ročník

31. 12. 2024 se uzavřel 10. ročník projektu Sokolská kapka krve. 

Darování krve je celospolečensky navýsost záslužná činnost, kterou se Sokol rozhodl dlouhodobě podporovat. Cílem projektu Sokolská kapka krve je především rozšířit povědomí mezi sokolskou i nesokolskou veřejností o~nutnosti krev darovat a získat nové a hlavně pravidelné dárce.

V~rámci tohoto projektu sokolské jednoty hlásí počty dárců krve a počty odběrů za jednotlivá pololetí; na konci roku se výsledky sečtou a zveřejní. Každý dárce pak obdrží odznak a přední jednoty pak věcné odměny. 

První ročník proběhl v~roce 2015: zúčastnilo se 12 jednot, kdy 36 dárců absolvovalo 59 odběrů. Od té doby počet dárců i odběrů vytrvale stoupá, zapojují se další nové jednoty. A~daří se také lákat nové dárce: těch letos bylo 61.

Celkově se v~roce 2024 projektu Sokolská kapka krve zúčastnilo 53 jednot a v~nich 357 dárců absolvovalo 866 odběrů. Vítězem se pak stal, jako od začátku vždy, Sokol Komárov, kde 40 dárců absolvovalo 119 odběrů.

Od počátku projektu tak sokolové darovali přes 2500 litrů krve.

Sokol Libeň se v~roce 2024 umístil na 18. místě, kdy 8 dárců absolvovalo celkem 16 odběrů. 

V~roce 2024 v~Sokole Libeň krev darovali: F. Dostál, T. Dragoun, M. Frost, V. Jakoubek, M. Kubů, B. Musilová, R. Zdvořilý a L. Zubáková.

{\sloppy
Krevní převody jsou ve zdravotnictví potřeba stále a nikdo neví, kdy sám bude krev potřebovat. Proto kdo můžete, alespoň to zkuste! Všechny dotazy též rád zodpovím na e-mailu: vit.jakoubek@sokol-liben.cz. Počty odběrů za 1. pololetí roku 2025 mi zasílejte na výše uvedený e-mail do 31. 7. 2025.}

\signature{Vít Jakoubek}{zdravotník jednoty a koordinátor projektu}
\vspace*{24pt}

% zadní obálka
\clearpage

\pagestyle{blank}
\newgeometry{margin=1cm}

\vspace*{96pt}

\pagecolor{sokolred}
\color{white}

\noindent {\fontsize{48pt}{56pt}\tyrs
se sokolským

\vspace*{24pt}

\noindent nazdar!}

\vspace*{\fill}

% \color{black}
\begin{center}
Vydává Tělocvičná jednota Sokol Libeň, Zenklova 37, Praha 8

\vspace*{12pt}

Na přípravě tohoto čísla se spolu s~autory jednotlivých textů podíleli:

grafická úprava – Martin Burian | jazyková úprava – Martina Waclawičová \\ editoři textů – Vít Jakoubek, Jan Přech
\end{center}

\end{document}