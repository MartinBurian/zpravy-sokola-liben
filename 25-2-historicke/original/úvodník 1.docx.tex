Vážené sestry, vážení bratři, milí členové a příznivci Sokola Libeň,

toto číslo Zpráv je jiné, než dosud znáte. Neobsahuje totiž aktuální
novinky a plány do budoucna, ale novinky a plány, z nichž některé byly
aktuální téměř před 100 lety.

V loňském roce naše Zprávy oslavily 50. ročník, a proto jsme se s
vzdělavatelkou Ankou Holanovou rozhodli si předchozí ročníky trochu
připomenout.

Anka vybrala několik svázaných ročníků, z nichž jsem pak vybíral články,
které jsem pak naskenoval.

Výběr to tedy není rozhodně reprezentativní. Nicméně snažil jsem se
vybrat jak texty k zamyšlení, tak texty veselé či jen ``nudná'' provozní
oznámení.

Texty jsou řazeny chronologicky, bez dalších komentářů. Výsledný dojem
je tedy na každém z vás.

Když jsem Zprávy pročítal, postupně se přede mnou otevírala historie
jednoty, jak jsem ji dosud neznal. Známé prostory v sokolovně pro mne
najednou ožívaly novým životem. Věřím, že na vás následující stránky
budou působit podobně.

Zprávy poprvé vyšly v roce 1927 a měly být spojovacím článkem mezi všemi
členy jednoty, kterých bylo přes 700 (dospělých!). Byly zdarma a
posílaly se poštou či je dobrovolníci roznášely do schránek. Vždy
obsahovaly zápisy ze schůzí Správního výboru, pozvánky na akce, zprávy z
výletů, inzeráty a mnoho dalšího.

Mnoho vážnějších textů pak dle mého názoru má sílu oslovit čtenáře i
dnes svou nečekanou aktuálností (z čehož jsem, v některých případech u
textů z roku 1937--1938, pociťoval občas lehké mrazení v zádech).

Jak léta postupovala, bylo vidět, jak se Zprávy stávaly obsažnější,
postupně přibývaly i pravidelné přílohy jako zprávy oddílu házené či
Hlídka dorostu. Byl také znát odlišný rukopis redaktorů.

Po revoluci v roce 1989 Zprávy začaly vycházet znovu a jsou téměř
kompletně k dispozici na internetových stránkách jednoty -- proto jsem z
nich vybral jen jednu ukázku.

Zájemci o další texty či studium historie naší jednoty nechť se s
důvěrou obrátí na naši vzdělavatelku Anku Holanovou.

Přeji vám všem inspirativní čtení!

Vít Jakoubek, editor Zpráv
