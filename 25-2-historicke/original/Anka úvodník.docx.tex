Milé čtenářstvo,

musím na úvod trochu poopravit Vítkův úvodník. Nápad sestavit
vzpomínkové číslo Zpráv jednoty byl jeho a já pouze vyštrachala v
archivu krabici se svázanými ročníky, které se tolik nesypou jako
nesvázané, mnohokrát otvírané sešitky z tenkého papíru a které jsme
potom částečně společně během výborových schůzí, částečně každý sám
pročítali, nadšeně píšíce tomu druhému pokaždé, když jsme objevili známé
jméno, nebo si posílali úryvky veselých článků. Ale nápad to byl čistě
Vítkův a finální výběr článků k otištění též.

Mnoho zpráv je suchých a provozních: počty cvičících, kolik se utratilo
za který podnik, že se v neděli jde na výlet a který cvičitel vede které
družstvo. Ale v těch druhých, jiných zprávách se před námi, jako když se
uzlíkují barevné bavlnky do náramku, splétal z jednotlivých nitek příběh
jednoty. Slova našich předchůdců zvěčněná ve čtvrtletníku nejsou jen
věcné proslovy nebo poučování, jak se má správný sokol chovat, ale
přečetli jsme si například přání ve formě básní, která si navzájem
skládali k narozeninám; komický návod, jak jednoduše udělat stojku na
bradlech; i to, jak v dobách hospodářské krize Sokol organizoval pomoc
těm, kteří přišli o zaměstnání. Zajímavý je popis rekonstrukce sokolovny
na konci dvacátých let -- téměř všechnu práci zastaly firmy našich
členů! Pro mě osobně byl nejsilnější zážitek číst o pohřbu bratra Vojty
Štekra, jehož čapka a ony kruhy, z nichž se zřítil při cvičení, byly
uloženy ve vitríně s jeho jmenovkou ve sborovně, a najednou jsem měla
před očima celý příběh a onen anonymní cvičitel dostal tvář a charakter,
když mi oči letěly po řádcích pohřební řeči bratra Štrosse.

„A je zvláštní náhodou, že kamenné poprsí br. Filipa nad jeho hrobem na
hřbitově libeňském vzhlíží svojí kamennou tváří ku hrobu bratra
Štekra\ldots`` A ve větvích stromu, který se nad hrobem sklání, se
skrývá a Vojtův klid stráží kamenný sokol sedící na náhrobním kameni. Už
95 let\ldots{}

Letos na Památný den sokolstva poneseme na korábský hřbitov o svíčku
víc.

Z každého řádku je vidět, jak tady ti lidé byli doma, jak strašně moc se
měli rádi a vzájemně si sebe vážili. A já jen doufám, že i naše
generace, po téměř stu let, takové vztahy mezi sebou i k jednotě má
taky, i když to už dáváme najevo jinak. Že si vzájemně pomáháme,
pracujeme pro Sokol jako pro společnou věc, jsme féroví a zodpovědní,
chceme druhé potěšit a pobavit, a i když se někdy škorpíme, máme se
rádi.

Kéž jsou vám texty našich předchůdců potěšením, pobavením, poučením i
inspirací.

Vaše vzdělavatelka Anka
